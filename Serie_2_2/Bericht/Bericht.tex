%% Copyright 2018 H.\ Rabus
%
% This work may be distributed and/or modified under the
% conditions of the LaTeX Project Public License, either version 1.3
% of this license or (at your option) any later version.
% The latest version of this license is in
%   http://www.latex-project.org/lppl.txt
% and version 1.3 or later is part of all distributions of LaTeX
% version 2005/12/01 or later.
%
% This work has the LPPL maintenance status `author-maintained'.
%
% This work consists of the file texbsp.tex
%

\documentclass[smallheadings]{scrartcl}

%%% GENERAL PACKAGES %%%%%%%%%%%%%%%%%%%%%%%%%%%%%%%%%%%%%%%%%%%%%%%%%%%%%%%%%%
% inputenc allows the usage of non-ascii characters in the LaTeX source code
\usepackage[utf8]{inputenc}
\usepackage{graphicx} 
\usepackage{float}
%\graphicspath{ {/u/hnatiuka/Praktikum/PPI/} }



% title of the document
\title{Bericht zu Serie 2}
% optional subtitle
%\subtitle{Draft from~\today}
% information about the author
\author{%
  Arsen Hnatiuk,\\%
  Max Huneshagen 
}
\date{\today} 


%%% LANGUAGE %%%%%%%%%%%%%%%%%%%%%%%%%%%%%%%%%%%%%%%%%%%%%%%%%%%%%%%%%%%%%%%%%%
% babel provides hyphenation patterns and translations of keywords like 'table
% of contents'
\usepackage[ngerman]{babel}

%%% HYPERLINKS %%%%%%%%%%%%%%%%%%%%%%%%%%%%%%%%%%%%%%%%%%%%%%%%%%%%%%%%%%%%%%%%
% automatic generation of hyperlinks for references and URIs
\usepackage{hyperref}

%%% MATH %%%%%%%%%%%%%%%%%%%%%%%%%%%%%%%%%%%%%%%%%%%%%%%%%%%%%%%%%%%%%%%%%%%%%%
% amsmath provides commands for type-setting mathematical formulas
\usepackage{amsmath}
\numberwithin{equation}{section}
% amssymb provides additional symbols
\usepackage{amssymb}
% HINT
% Use http://detexify.kirelabs.org/classify.html to find unknown symbols!

%%% COLORS %%%%%%%%%%%%%%%%%%%%%%%%%%%%%%%%%%%%%%%%%%%%%%%%%%%%%%%%%%%%%%%%%%%%
% define own colors and use colored text
\usepackage[pdftex,svgnames,hyperref]{xcolor}

%%% Code Listings %%%%%%%%%%%%%%%%
% provides commands for including code (python, latex, ...)
\usepackage{listings}
\definecolor{keywords}{RGB}{255,0,90}
\definecolor{comments}{RGB}{0,0,113}
\definecolor{red}{RGB}{160,0,0}
\definecolor{green}{RGB}{0,150,0}
\lstset{language=Python, 
        basicstyle=\ttfamily\small, 
        keywordstyle=\color{keywords},
        commentstyle=\color{comments},
        stringstyle=\color{red},
        showstringspaces=false,
        identifierstyle=\color{green},
        }


\usepackage{paralist}
\usepackage{nicefrac}
% setting the font style for input und returns in description items
\newcommand{\initem}[2]{\item[\hspace{0.5em} {\normalfont\ttfamily{#1}} {\normalfont\itshape{(#2)}}]}
\newcommand{\outitem}[1]{\item[\hspace{0.5em} \normalfont\itshape{(#1)}]}
\newcommand{\bfpara}[1]{
	
	\noindent \textbf{#1:}\,}


\begin{document}

% generating the title page
\maketitle
% generating the table of contents (requires to run pdflatex twice!)
\tableofcontents
\bigskip

\hrule
\hrule

%%% BEGIN OF CONTENT %%%%%%%%%%%%%%%%%%%%%%%%%%%%%%%%%%%%%%%%%%%%%%%%%%%%%%%%%%

\section{Einleitung}

In diesem Bericht werden Matrizen betrachtet, die dem numerischen Lösen des Randwertproblems 
\begin{align}
\begin{split}
-\Delta u&=f(x)\text{ in }\Omega=(0, 1)^d, d\in\{1,2,3\}\\
u \vert _{\partial\Omega}&=u_D\text{ für ein gegebenes }u_D.
\end{split}
\label{eq:rwp}
\end{align}
Hierbei bezeichnet 
\begin{align}
\Delta :=\sum_{j=1}^{d}\frac{\partial^2}{\partial x_j^2}
\end{align}
den Laplace-Operator. Zur numerischen Lösung ist es notwendig, eine Diskretisierung von $\Omega$ durchzuführen, die es erlaubt \eqref{eq:rwp} in ein Problem der Form 
\begin{align}
A^{(d)}\hat{u}=b
\label{eq:lgs}
\end{align}
mit $A^{(d)}\in\mathbb{R}^{(n-1)^d\times(n-1)^d}$, $\hat{u},b\in\mathbb{R}^{n-1}$ und der Diskretisierung $n\in\mathbb{N}$ zu überführen.
Die Gestalt dieser Matrizen $A^{(d)}$ ist in der Schnittstellendoke von \textit{sparse.py} bereits beschrieben worden und wird hier im Anhang wiederholt; weiter unten soll gezeigt werden, dass die dort definierten Matrizen tatsächlich dem diskretisierten Laplace-Operator entsprechen. Ebenso wird der Speicherplatzbedarf der verwendeten \textit{scipy.sparse.dok\_matrix}-Objekt analytisch und experimentell untersucht und dem von \glqq gewöhnlichen\grqq vollbesetzten Matrizen entgegen gehalten.
\section{Theorie}

\subsection{Speicherplatzbedarf}
\label{sec:bedarf}
Der wesentliche Unterschied zwischen vollbesetzten und \textit{sparse}-Matrizen  liegt in ihrem Speicherbedarf. Während eine vollbesetzte Matrix alle Einträge wie in einem Array speichert, werden nur die nicht-Null Einträge in einer \textit{sparse}-Matrix gespeichert. Also braucht ein \textit{numpy.matrix}-Objekt der Dimension $(n \times n)$ gerade $n^2$ \textit{float}-Speicherplätze (jeweils 64 bit). Ein \textit{scipy.sparse.dok\_matrix}-Objekt  ist ein Wörterbuch-Objekt, dessen Schlüssel die \textit{(Zeile, Spalte)}-Koordinaten und die Werte an diesen Stellen sind. 
Also brauchen $n$ Nicht-Null Einträge $2n$ \textit{int}-Speicherplätze (für die zwei Schüssel-\textit{arrays}), jeweils 24 bit, und $n$ \textit{float}-Speicherplätze für den Matrixeintrag. Dagegen sind keine Null Einträge gespeichert, also enthält das Wörterbuch die Schlüssel mit den Koordinaten der Null-Einträge. Die Dimension der Matrix wird durch zwei \textit{int}-Objekte gespeichert, deren Größe hier vernachlässigt werden kann. Wir wenden nun die Speicherbedarfsanalyse auf eine Matrix wie in der Aufgabestellung mit Feinheit der Diskretisierung $n$ an. Wir approximieren dabei der Übersicht halber $2 \cdot 24 \approx 64$, um einen Wörterbuch-Eintrag als zwei \textit{float}-Einträge zu betrachten, damit nur \textit{float}-Speicherplätze berücksichtigt werden müssen.

\paragraph{d = 1}
In der ersten Dimension hat die vollbesetzte Matrix $n^2$ Einträge, also einen Speicherbedarf der Ordnung $\mathcal{O}(n^2)$.

Die Nicht-Null-Einträge dieser Matrix stehen auf den drei zentralen Diagonalen (Tridiagonalmatrix). Also benötigt das \textit{scipy.sparse.dok\_matrix}-Objekt $2(n+2(n-1)) = \mathcal{O}(n)$ Speicherplätze.

\paragraph{d = 2}
Die $d=2$-Matrix besteht aus $n^2$ Matrizen, die die gleiche Größe wie eine $d=1$-Matrix haben. Also ist der Speicherbedarf der \textit{numpy}-Matrix der Ordnung $\mathcal{O}(n^4)$.

 Diese Matrix hat $n$ viele $d=1$-Matrizen auf der Hauptdiagonale  und $2(n-1)$ Einheitsmatrizen auf den zwei Nebendiagonalen, mit jeweils $n$ Nicht-Null Einträgen. Insgesamt ergibt dies $2(n(\mathcal{O}(6n))+2n(n-1))=\mathcal{O}(n^2)$ Speicherplätze.
 
 \paragraph{d = 3}
 Die $d=3$-Matrix besteht aus $n^2$ Matrizen, die die gleiche Größe wie eine $d=2$-Matrix haben. Also ist der Speicherbedarf der \textit{numpy}-Matrix der Ordnung $\mathcal{O}(n^6)$.
 
 Die untersuchte Matrix hat $n$ viele $d=2$-Matrizen auf der Hauptdiagonale  und $2(n-1)$ Einheitsmatrizen auf den zwei Nebendiagonalen, mit jeweils $n^2$ nicht-Null Einträge. Insgesamt macht das $2(n(\mathcal{O}(16n^2))+2n^2(n-1))=\mathcal{O}(n^3)$ Speicherplätze. 
Zusammenfassend ergibt sich also als Quotient der Speicherbedarfe:
\begin{align}
a_{\text{nn},d}:=\frac{\text{Speicherbedarf als \textit{sparse}-Matrix}}{\text{Speicherbedarf als vollbesetzte Matrix}}=\mathcal{O}(n^{-d})
\label{eq:quot}
\end{align}
Die Zahl $a_{\text{nn},d}$ entspricht aus o.\,g. Gründen der relativen Anzahl von Nicht-Null-Einträgen der Matrix $A^{(d)}$ und wird in Abschnitt \ref{sec:experimente} experimentell untersucht.

\subsection{Lineares Gleichungssystem}

In diesem Abschnitt soll die Gestalt der Matrix $A^{(d)}$ herlgeleitet werden, mit deren Hilfe \eqref{eq:rwp} in \eqref{eq:lgs} überführt werden kann. Dazu muss $\Omega$ diskretisiert werden.


\paragraph{ Dsikretisierung}

Wir Zerlegen den Definitionsbereich mithilfe eines uniformen Gitters mit Schrittweite $h = 1/n$. Die Schnittpunkte auf diesem Gitter sind dann die Punkte, auf den die Funktion $u$ zu approximieren ist. Sei $P$ die Menge dieser Punkte. Für $d=3$ hat jeder Punkt $x\in P$ die Gestalt $x = (\frac{i}{n}, \frac{j}{n}, \frac{k}{n})$ für natürliche Zahlen $i, j, k \leq n$. Entsprechend gilt für $d=2$ $x=(\frac{i}{n}, \frac{j}{n})$ und für $d=1$ $x=(\frac{i}{n})$.
Um diese $P$ als Spaltenvektor wie in \eqref{eq:lgs} darstellen zu können, muss zunächst eine Nummerierung aller Elemente von $P$ mit nur einem Index erfolgen. Für $d=1$ ist dies trivial: $x_i = (\frac{i}{n})$. 
Für $d=2$ soll zur Nummerierung zeilenweise vorgegangen werden, so wie in Abb.~\ref{im:gitter_2d} angedeutet.

\begin{figure}[H]
\includegraphics[width=\textwidth]{Bilder/gitter_d2}
\caption{Nummerierung der Gitterpunkte für $d=2$ (Quelle: Aufgabenstellung).}
\label{im:gitter_2d}
\end{figure}

Im Falle $d=3$ wird das dann dreidimensionale Gitter ganz analog durchummeriert, indem zuerst die $k=1$-Ebene wie im Fall $d=2$ durchgegangen wird, dann die $k=2$-Ebene usw. Zur Formalisierung dessen wird die folgende Funktion eingeführt:
\begin{align}
g(m)=
\begin{cases} 
n-1 &m\mod{n-1}=0 \\
m\mod{n-1} & \text{sonst} \\
\end{cases}.
\end{align}
Der $m$-te Eintrag von $x$ kodiert dann die folgenden Koordiaten :
\begin{align}
x_m=
\begin{cases}
\left(\frac{g(m)}{n},\frac{\ulcorner\frac{m}{n}\urcorner}{n} \right) 
& \text{für } d=2\\ 
\left(\frac{g(m)}{n}, \frac{g(\ulcorner\frac{m}{n}\urcorner)}{n}, \frac{\ulcorner\frac{m}{(n-1)^2+1}\urcorner}{n}\right) 
& \text{für } d=3
\end{cases}
\label{eq:code}
\end{align}
wobei $\ulcorner \cdot \urcorner$ die Aufrundung bezeichnet. Nach dieser Rechenvorschrift wird jedem $m$ zwischen 1 und $(n-1)^d$ ein eindeutiges Punkt zugewiesen. Man sieht auch: für $x_m = (\frac{i}{n}, \frac{j}{n}, \frac{k}{n})$ gilt $x_{m\pm(n-1)} = (\frac{i}{n}, \frac{j\pm1}{n}, \frac{k}{n})$ und $x_{m\pm(n-1)^2} = (\frac{i}{n}, \frac{j}{n}, \frac{k\pm1}{n})$.

\paragraph{Diskretisierung der Differentialgleichung}
Die gefundene Umwandlung der Elemente von $P$ in einen Spaltenvektor soll nun genutzt werden, um \eqref{eq:rwp} zu diskretiisieren. Dazu wird zunächst die folgende Formel aus Serie~1 rekapituliert, die direkt aus der Taylor-Entwicklung folgt: 
\begin{align}
\frac{\partial^2 u}{\partial x_j^2}(x)=\frac{u(x-e_jh)+u(x+e_jh)-2u(x)}{h^2}+\mathcal{O}(h^2)
\label{eq:diff}
\end{align} 

Daraus folgt mit \eqref{eq:rwp} unter Vernachlässigung von $\mathcal{O}(h^2)$ für $d=3$:
\begin{align}
f(x) = \frac{-u(x-e_1h)-u(x+e_1h)-u(x-e_2h)+u(x+e_2h)-u(x-e_3h)+u(x+e_3h)+2du(x)}{h^2}.
\end{align}
Die Fälle $d\in\{1,2\}$ lassen sich hieraus einfach ableiten, indem die \glqq überflüssigen\grqq Einhaitsvektoren durch $0$ ersetzt werden.
%Wenn wir $h$ wie im letzten Absatz auffassen, können wir die Formel bezüglich der Punkte in $P$ mit der Eigenschaft $x_m=(\frac{i}{n}, \frac{j}{n}, \frac{k}{n})$, $i, j, k \notin \{1, (n-1)\}$ (sei $P^*$ die Menge dieser Punkte) und die darauf definierte Nummerierung darstellen:
Es tragen also für jeden Punkt nur die nächsten Nachbarn im Abstand $h$ zum Wert von $u$ an dieser Stelle bei. Auf dem quadratischen ($d=2$) bzw. kubischen ($d=3$) Gitter sind die nächsten Nachbarn von $x_m$ aber nicht benachbart der oben etablierten Vektor-Kodierung \eqref{eq:code}. Dies führt dazu, dass $A^{(d)}$ für $d\neq 1$ nicht länger eine simple Tridiagonalgestalt hat. Um diese Form zu erhalten, wird zunähchst foldende Menge definiert:
\begin{align}
P^*:=\lbrace x_m=(\frac{i}{n}, \frac{j}{n}, \frac{k}{n}), i, j, k \notin \lbrace 1, (n-1)\rbrace\rbrace.
\end{align}
Somit gilt für alle $x_m\in P^*$:
\begin{align}
f(x_m) = \frac{-\hat{u}(x_{m-1})-\hat{u}(x_{m+1})-\hat{u}(x_{m-(n-1)})+\hat{u}(x_{m+(n-1)})-\hat{u}(x_{m-(n-1)^2})+\hat{u}(x_{m+(n-1)^2})+2d\hat{u}(x)}{h^2},
\label{eq:diff2}
\end{align}
wobei $\hat{u}$ eine Approximation von $u$ ist (wir haben einen Fehler der Ordnung $\mathcal{O}(h^2)$ vernachlässigt), und die Terme $x_{m\pm(n-1)}$ und $x_{m\pm(n-1)^2}$ für $d=1$ und $x_{m\pm(n-1)^2}$ für $d=2$	verschwinden. Aus der Definition der gewählten Nummerierung sieht man, dass die Formel \eqref{eq:diff2} dem linearen Gleichungssystem $A_{P^*}\hat{u}=b_{P^*}$ entspricht, was im Wesentlichen \eqref{eq:lgs} entspricht, allerdings ohne die Zeilen, die den Punkten in $P\setminus P^*$ entsprechen.

\paragraph{Aufstellen des linearen Gleichungssystems}

Damit wir das lineare Gleichungssystem für alle $x_m$ aufstellen können, müssen die Randbedingungen $u_D \in \partial\Omega$ berücksichtigt werden, weil die Punkte $x \pm e_jh$ nicht immer in $P$ liegen. In den Matrizen aus Teil 1 werden diese Einträge nicht berücksichtigt. Um zur Darstellung \eqref{eq:lgs} zu gelangen, müssen wir $b_P$ (der Vektor mit Einträgen $f(x_m)$ für $m \in P$) mit den Termen aus dem Rand $\partial\Omega$ vervollständigen. 
Sei $b_i :=f(x_i)$. Wir erweitern nun diese Werte wie folgt:

\begin{itemize}

\item \textbf{für $\boldsymbol{d\in \{1, 2, 3\} }$:}
Für $m = k(n-1)+1$ mit ein $k \in \mathbb{N}\cup \{0\}$ setze man $b_m = b_m + \frac{u(x_m - e_ih)}{h^2}$. Für $m = k(n-1) +1$ mit ein $k \in \mathbb{N}$ setze man $b_m = b_m + \frac{u(x_m + e_ih)}{h^2}$. 
\item \textbf{für $\boldsymbol{d\in \{2, 3\} }$:}
Für $m \in (k(n-1)^2+1, k(n-1)^2+n-1)$ mit  $k \in \mathbb{N}\cup \{0\}$ setze  $b_m = b_m + \frac{u(x_m)-e_jh}{h^2}$. Für $m \in (k(n-1)^2-(n-1), k(n-1)^2-1)$ mit $k \in \mathbb{N}$ setze  $b_m = b_m + \frac{u(x_m)+e_jh}{h^2}$.
\item \textbf{für $\boldsymbol{d=3 }$:}
Für $m\in (1, (n-1)^2)$ setze man $b_m = b_m + \frac{u(x_m-e_kh)}{h^2}$. Für $m\in ((n-1)^3-(n-2), (n-1)^3)$ setze man $b_m = b_m + \frac{u(x_m+e_kh)}{h^2}$.  
\end{itemize}
Durch diese Umformungen geht die resultierende Matrix von $A_{P^*}$ zu $A^{(d)}$ über, also folgt
\begin{align}
A^{(d)}\hat{u}=b
\end{align}
mit den in der Schnittstellendokumentation sowie in Abschnitt~\ref{sec:anhang} beschriebenen $A^{(d)}$.

\section{Experimente}
\label{sec:experimente}
In \textit{experimente.py} wird die relative Anzahl der Nicht-Null-Einträge der Matrix $A^{(d)}$ für $d=1,2,3$ numerisch untersucht und anschließend graphisch dargestellt. Darüber hinaus wird die Gesamtanzahl der Einträge einer vollbesetzen Matrix der gleichen Größe wie $A^{(d)}$ in Abhängigkeit von $n$ geplottet. Diese ergibt sich zu $(n-1)^{2d}$, da $A^{(d)}\in\mathbb{R}^{(n-1)^d\times(n-1)^d}$ Das Ergebnis des Experiments ist in Abb.~\ref{im:nn_eintr} doppeltlogarithmisch dargestellt.

\begin{figure}[H]
\includegraphics[width=\textwidth]{Bilder/nn_eintraege}

\caption{Durchgezogen: Nicht-Null-Einträge von $A^{(d)}$ für $d=1,2,3$ (relative Anzahl). Gestrichelt: Einträge einer vollbesetzten Matrix der gleichen Größe wie $A^{(d)}$.}
\label{im:nn_eintr}
\end{figure}



Man erkennt, dass die relative Anzahl der Nicht-Null-Einträge (und damit der Speicheraufwand des \textit{scipy.dok\_matrix}-Objektes für jedes untersuchte $d$) verschieden stark abfällt. Bezeichnet $a_{\text{nn},d}(n)$ die Anzahl der 
Nicht-Null-Einträge von $A^{(d)}$, so ergibt sich durch eine graphische Analyse, dass $a_{\text{nn},d}\sim n^{-d}$ in Übereinstimmung mit \eqref{eq:quot}. Mit zunehmender Feinheit der Diskretisierung wird die Matrix deutlich dünner besetzt und die Speicherung als Sparse-Matrix immer sinnvoller. Man beachte dazu insbesondere die zunehmende Differenz zum Wert 1, der in Abb.~\ref{im:nn_eintr} durch eine gepunktete Linie eingetragen wurde. 
Um die Rechenzeit sinnvoll zu beschränken, kann $n$ nicht beliebig groß gewählt werden die Wahl des maximalen $n$ hängt dabei wegen $\dim(A^{(d)})=(n-1)^2$ von $d$ ab. Die Werte für $n$ wurden in \textit{experimente.py} so gewählt, dass die Rechenzeit einige Sekunden nicht übersteigt (und jeweils auf den nächsten \textit{int}-Wert abgerundet):
\begin{figure}[H]
\centering
\begin{tabular}{|c|c|}
\hline
$\boldsymbol{d}$ & $\boldsymbol{\log_{10}(n_{max})}$\\
\hline
1 & 4\\
\hline
2 & 2\\
\hline
3 & 1,3\\
\hline

\end{tabular}
\caption{Maximale untersuchte Werte für $n$ in Abhängigkeit von $d$}
\end{figure}
Durch die Verwendung der \textit{sparse}-Matrizen wäre ein die Untersuchung von größeren $n$ mit mehr Rechenzeit ebenfalls möglich, ohne einen \textit{MemoryError} zu provozieren, worauf hier aber verzichtet wird.

\section{Zusammenfassung}

In diesem Bericht wurden die zur numerischen Lösung der Poisson-Gleichung nötigen Matrizen $A^{(d)}$ untersucht. Der Speicherbedarf dieser Matrizen als \textit{sparse}-Matrix wurde analytisch und experimentell in Abhängigkeit von $d$ und $n$ ermittelt und dem einer gleichgroßen vollbesetzten Matrix gegenübergestellt. Dabei wurde deutlich, dass die $A^{(d)}$ für angemessen große $n$ sehr dünn besetzte Matrizen sind und sich daher eine Speicherung als \textit{sparse}-Matrizen lohnt. 

Es wurde gezeigt, dass $A^{(d)}$ tatsächlich dem diskretisierten Laplace-Operator für $d$ Dimensionen entspricht und sich damit die Poisson-Gleichung als lineares Gleichhungssystem schreiben lässt. 

\cleardoublepage
\appendix
\section{Anhang}\label{sec:anhang}
\emph{An dieser Stelle soll die Gestalt der Matrizen $A^{(d)}$ wiederholt werden. Diese Beschreibung befindet sich bereits in der Schnittstellendokumentation der Klasse \emph{sparse.Sparse} und wird hier lediglich zur besseren Übersicht erneut aufgeführt.}\newline
\newline
Zunächst definiert man für $d=1$:
\begin{align}
A^{(1)}:= 
\begin{pmatrix}
2 & -1 & 0 & \dots & 0\\
-1 & 2 & -1 & \dots & 0\\
\vdots & \ddots & \ddots & \ddots & \vdots\\
0 & 0 & -1 & 2 & -1\\
0 & 0 & \dots & -1 & 2
\end{pmatrix}\in \mathbb{R}^{(n-1)\times(n-1)}.
\label{eq:a_1_def}
\end{align}
Definiert man mit Hilfe der $(l\times l)$-Einheitsmatrix $I^l\in \mathbb{R}^{l\times l}$ die folgenden Matrizen:
\begin{align}
\mathcal{I}_d&:=I_{(n-1)^d}\in\mathbb{R}^{(n-1)^d\times(n-1)^d}\\
\mathcal{A}_1(k)&:=
\begin{pmatrix}
2k & -1 & 0 & \dots & 0\\
-1 & 2k & -1 & \dots & 0\\
\vdots & \ddots & \ddots & \ddots & \vdots \\
0 & 0 & -1 & 2k & -1\\
0 & 0 & \dots & -1 & 2k\\
\end{pmatrix} \in \mathbb{R}^{(n-1)\times(n-1)},
\end{align}
so lässt sich die folgende rekursive Matrixfolge definieren:
\begin{align}
\mathcal{A}_l(k):=
\begin{pmatrix}
\mathcal{A}_{l-1}(k) & -\mathcal{I}_{l-1} & 0 & 0 & \dots & 0\\
-\mathcal{I}_{l-1} & \mathcal{A}_{l-1}(k) & -\mathcal{I}_{l-1} & 0 & \dots & 0\\
0 & \ddots & \ddots & \ddots & \ddots & \vdots \\
\vdots & \ddots & \ddots & \ddots & \ddots & \vdots \\
0 & \dots & 0 & -\mathcal{I}_{l-1} & \mathcal{A}_{l-1}(k) & -\mathcal{I}_{l-1} \\
0 & \dots & 0 & 0 & -\mathcal{I}_{l-1} & \mathcal{A}_{l-1}(k)
\end{pmatrix} \in \mathbb{R}^{(n-1)^l\times(n-1)^l}.
\end{align}
Die Matrizen $A^{(d)}$ ergeben sich für den Fall $l=k=d$:

\begin{align}
A^{(1)}&=\mathcal{A}_1(1)\text{~~~~~(in Übereinstimmung mit \eqref{eq:a_1_def})}\\
A^{(2)}&:=\mathcal{A}_2(2)\\
A^{(3)}&:=\mathcal{A}_3(3).
\end{align}

\begin{thebibliography}{9}
\bibitem{wiki} Kein Autor, Aufgerufen am 22.11.2018, \textit{Sparse Matrix}. 
\url{https://en.wikipedia.org/wiki/Sparse_matrix}
\end{thebibliography}


%%% END OF DOCUMENT %%%%%%%%%%%%%%%%%%%%%%%%%%%%%%%%%%%%%%%%%%%%%%%%%%%%%%%%%%%
\end{document}
