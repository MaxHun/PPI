%% Generated by Sphinx.
\def\sphinxdocclass{report}
\documentclass[letterpaper,10pt,ngerman]{sphinxmanual}
\ifdefined\pdfpxdimen
   \let\sphinxpxdimen\pdfpxdimen\else\newdimen\sphinxpxdimen
\fi \sphinxpxdimen=.75bp\relax

\usepackage[utf8]{inputenc}
\ifdefined\DeclareUnicodeCharacter
 \ifdefined\DeclareUnicodeCharacterAsOptional
  \DeclareUnicodeCharacter{"00A0}{\nobreakspace}
  \DeclareUnicodeCharacter{"2500}{\sphinxunichar{2500}}
  \DeclareUnicodeCharacter{"2502}{\sphinxunichar{2502}}
  \DeclareUnicodeCharacter{"2514}{\sphinxunichar{2514}}
  \DeclareUnicodeCharacter{"251C}{\sphinxunichar{251C}}
  \DeclareUnicodeCharacter{"2572}{\textbackslash}
 \else
  \DeclareUnicodeCharacter{00A0}{\nobreakspace}
  \DeclareUnicodeCharacter{2500}{\sphinxunichar{2500}}
  \DeclareUnicodeCharacter{2502}{\sphinxunichar{2502}}
  \DeclareUnicodeCharacter{2514}{\sphinxunichar{2514}}
  \DeclareUnicodeCharacter{251C}{\sphinxunichar{251C}}
  \DeclareUnicodeCharacter{2572}{\textbackslash}
 \fi
\fi
\usepackage{cmap}
\usepackage[T1]{fontenc}
\usepackage{amsmath,amssymb,amstext}
\usepackage{babel}
\usepackage{times}
\usepackage[Sonny]{fncychap}
\usepackage[dontkeepoldnames]{sphinx}

\usepackage{geometry}

% Include hyperref last.
\usepackage{hyperref}
% Fix anchor placement for figures with captions.
\usepackage{hypcap}% it must be loaded after hyperref.
% Set up styles of URL: it should be placed after hyperref.
\urlstyle{same}
\addto\captionsngerman{\renewcommand{\contentsname}{Contents:}}

\addto\captionsngerman{\renewcommand{\figurename}{Abb.}}
\addto\captionsngerman{\renewcommand{\tablename}{Tab.}}
\addto\captionsngerman{\renewcommand{\literalblockname}{Quellcode}}

\addto\captionsngerman{\renewcommand{\literalblockcontinuedname}{continued from previous page}}
\addto\captionsngerman{\renewcommand{\literalblockcontinuesname}{continues on next page}}

\addto\extrasngerman{\def\pageautorefname{Seite}}

\setcounter{tocdepth}{1}



\title{Serie 3 Documentation}
\date{16.12.2018}
\release{}
\author{Arsen Hnatiuk, Max Huneshagen}
\newcommand{\sphinxlogo}{\vbox{}}
\renewcommand{\releasename}{Release}
\makeindex

\begin{document}
\ifnum\catcode`\"=\active\shorthandoff{"}\fi
\maketitle
\sphinxtableofcontents
\phantomsection\label{\detokenize{index::doc}}



\chapter{Die Sparse-Klasse}
\label{\detokenize{index:welcome-to-serie-3-s-documentation}}\label{\detokenize{index:module-sparse_erw}}\label{\detokenize{index:die-sparse-klasse}}\index{sparse\_erw (Modul)}
sparse.py stellt die Klasse Sparse zur Verfuegung, mit der die Matrix A\textasciicircum{}(d) fuer d=1,2,3
bestimmt und analysiert werden kann.
\index{Sparse (Klasse in sparse\_erw)}

\begin{fulllineitems}
\phantomsection\label{\detokenize{index:sparse_erw.Sparse}}\pysiglinewithargsret{\sphinxbfcode{class }\sphinxcode{sparse\_erw.}\sphinxbfcode{Sparse}}{\emph{dim}, \emph{dis}, \emph{r\_s=None}, \emph{ex\_lsg=None}}{}
Diese Klasse erlaubt das Erstellen der Matrizen A\textasciicircum{}(d) fuer d in {[}1,2,3{]}. Diese Matrizen werden
z. B. fuer die Berechnung der DGL u‘‚(x)=-f(x) verwendet. Es handelt sich bei diesen Matrizen
um sehr duenn besetzte Block-Band-Matrizen, was die Verwendung von sog. sparse-Matrizen
in der numerischen Umsetzung nahelegt.

Attribute:
\begin{quote}
\begin{description}
\item[{dim (int):}] \leavevmode
Raumdimension des zu untersuchenden Gebietes.

\item[{dis (numpy.ndarray aus floats):}] \leavevmode
Mass fuer die Diskretisierung des zu untersuchenden Gebietes.

\item[{matr (scipy.dok\_matrix-Objekt):}] \leavevmode
A\textasciicircum{}(d) mit Diskretisierung dis.

\end{description}
\end{quote}
\index{anz\_n\_abs() (Methode von sparse\_erw.Sparse)}

\begin{fulllineitems}
\phantomsection\label{\detokenize{index:sparse_erw.Sparse.anz_n_abs}}\pysiglinewithargsret{\sphinxbfcode{anz\_n\_abs}}{}{}
Gibt die Anzahl von Eintraegen von A\textasciicircum{}(d) zurueck, die gleich 0 sind.

Input: -
\begin{description}
\item[{Return:}] \leavevmode\begin{description}
\item[{(int):}] \leavevmode
Anzahl von Nulleintraegen von A\textasciicircum{}(d).

\end{description}

\end{description}

\end{fulllineitems}

\index{anz\_n\_lu\_abs() (Methode von sparse\_erw.Sparse)}

\begin{fulllineitems}
\phantomsection\label{\detokenize{index:sparse_erw.Sparse.anz_n_lu_abs}}\pysiglinewithargsret{\sphinxbfcode{anz\_n\_lu\_abs}}{}{}
Gibt die Anzahl von Eintraegen von L bzw. U zurueck, die gleich 0 sind.

Input: -
\begin{description}
\item[{Return:}] \leavevmode\begin{description}
\item[{(int-Tupel):}] \leavevmode
Anzahl von Nulleintraegen von L und U.

\end{description}

\end{description}

\end{fulllineitems}

\index{anz\_n\_lu\_rel() (Methode von sparse\_erw.Sparse)}

\begin{fulllineitems}
\phantomsection\label{\detokenize{index:sparse_erw.Sparse.anz_n_lu_rel}}\pysiglinewithargsret{\sphinxbfcode{anz\_n\_lu\_rel}}{}{}
Gibt die relative Anzahl von Eintraegen von L bzw. U zurueck, die gleich 0 sind.

Input: -
\begin{description}
\item[{Return:}] \leavevmode\begin{description}
\item[{(int-Tupel):}] \leavevmode
Relative Anzahl von Nulleintraegen von L und U.

\end{description}

\end{description}

\end{fulllineitems}

\index{anz\_n\_rel() (Methode von sparse\_erw.Sparse)}

\begin{fulllineitems}
\phantomsection\label{\detokenize{index:sparse_erw.Sparse.anz_n_rel}}\pysiglinewithargsret{\sphinxbfcode{anz\_n\_rel}}{}{}
Gibt die relative Anzahl von Eintraegen von A\textasciicircum{}(d) zurueck, die gleich 0 sind.

Input: -
\begin{description}
\item[{Return:}] \leavevmode\begin{description}
\item[{(int):}] \leavevmode
Relative Anzahl von Nulleintraegen von A\textasciicircum{}(d).

\end{description}

\end{description}

\end{fulllineitems}

\index{anz\_nn\_abs() (Methode von sparse\_erw.Sparse)}

\begin{fulllineitems}
\phantomsection\label{\detokenize{index:sparse_erw.Sparse.anz_nn_abs}}\pysiglinewithargsret{\sphinxbfcode{anz\_nn\_abs}}{}{}
Gibt die Anzahl von Eintraegen von A\textasciicircum{}(d) zurueck, die ungleich 0 sind.

Input: -
\begin{description}
\item[{Return:}] \leavevmode\begin{description}
\item[{(int):}] \leavevmode
Anzahl von Nicht-Nulleintraegen von A\textasciicircum{}(d).

\end{description}

\end{description}

\end{fulllineitems}

\index{anz\_nn\_lu\_abs() (Methode von sparse\_erw.Sparse)}

\begin{fulllineitems}
\phantomsection\label{\detokenize{index:sparse_erw.Sparse.anz_nn_lu_abs}}\pysiglinewithargsret{\sphinxbfcode{anz\_nn\_lu\_abs}}{}{}
Gibt die Anzahl von Eintraegen von L bzw. U zurueck, die ungleich 0 sind.

Input: -
\begin{description}
\item[{Return:}] \leavevmode\begin{description}
\item[{(int-Tupel):}] \leavevmode
Anzahl von Nicht-Nulleintraegen von L und U.

\end{description}

\end{description}

\end{fulllineitems}

\index{anz\_nn\_lu\_rel() (Methode von sparse\_erw.Sparse)}

\begin{fulllineitems}
\phantomsection\label{\detokenize{index:sparse_erw.Sparse.anz_nn_lu_rel}}\pysiglinewithargsret{\sphinxbfcode{anz\_nn\_lu\_rel}}{}{}
Gibt die relative Anzahl von Eintraegen von L bzw. U zurueck, die ungleich 0 sind.

Input: -
\begin{description}
\item[{Return:}] \leavevmode\begin{description}
\item[{(int-Tupel):}] \leavevmode
Relative Anzahl von Nulleintraegen von L und U.

\end{description}

\end{description}

\end{fulllineitems}

\index{anz\_nn\_rel() (Methode von sparse\_erw.Sparse)}

\begin{fulllineitems}
\phantomsection\label{\detokenize{index:sparse_erw.Sparse.anz_nn_rel}}\pysiglinewithargsret{\sphinxbfcode{anz\_nn\_rel}}{}{}
Gibt die relative Anzahl von Eintraegen von A\textasciicircum{}(d) zurueck, die ungleich 0 sind.

Input: -
\begin{description}
\item[{Return:}] \leavevmode\begin{description}
\item[{(int):}] \leavevmode
Relative Anzahl von Nulleintraegen von A\textasciicircum{}(d).

\end{description}

\end{description}

\end{fulllineitems}

\index{constr\_mat\_l\_k() (Methode von sparse\_erw.Sparse)}

\begin{fulllineitems}
\phantomsection\label{\detokenize{index:sparse_erw.Sparse.constr_mat_l_k}}\pysiglinewithargsret{\sphinxbfcode{constr\_mat\_l\_k}}{\emph{k}, \emph{dim}, \emph{dis}}{}
Konstruiert die Matrix A\_l(k) mit der gewuenschten Diskretisierung.

Input:
\begin{quote}
\begin{description}
\item[{k (float):}] \leavevmode
Bestimmt den Wert auf der Hauptdiagonalen der untersuchten Matrix (=2*k)

\item[{dim (int, moegliche Werte: 1, 2, 3):}] \leavevmode
Raumdimension des betrachteten Gebietes.

\item[{dis (int):}] \leavevmode
Diskretisierung des Gebietes.

\end{description}
\end{quote}
\begin{description}
\item[{Return:}] \leavevmode\begin{description}
\item[{(scipy.sparse.dok\_matrix-Objekt):}] \leavevmode
A\_l(k) mit der gewuenschten Diskretisierung.

\end{description}

\end{description}

\end{fulllineitems}

\index{kond\_a\_d\_zs() (Methode von sparse\_erw.Sparse)}

\begin{fulllineitems}
\phantomsection\label{\detokenize{index:sparse_erw.Sparse.kond_a_d_zs}}\pysiglinewithargsret{\sphinxbfcode{kond\_a\_d\_zs}}{}{}
Gibt die Kondition der Matrix A\textasciicircum{}(d) bezueglich der Zeilensummennorm zurueck.

Input: -
\begin{description}
\item[{Return:}] \leavevmode\begin{description}
\item[{(float):}] \leavevmode
Kondition von A\textasciicircum{}(d) bzgl. Zeilennorm.

\end{description}

\end{description}

\end{fulllineitems}

\index{l\_u\_zerl() (Methode von sparse\_erw.Sparse)}

\begin{fulllineitems}
\phantomsection\label{\detokenize{index:sparse_erw.Sparse.l_u_zerl}}\pysiglinewithargsret{\sphinxbfcode{l\_u\_zerl}}{}{}
Errechnet die L-U-Zerlegung von A\textasciicircum{}(d)=P\_r\textasciicircum{}T*L*U*P\_c\textasciicircum{}T

Input: -
\begin{description}
\item[{Return:}] \leavevmode\begin{description}
\item[{(list):}] \leavevmode
Nullter Eintrag: das mit der Zerlegung korrespondierende scipy.SuperLU-Objekt.
Erster Eintrag: Tripel aus den Matrizen P\_r, P\_c, L, U.

\end{description}

\end{description}

\end{fulllineitems}

\index{lgs\_lsg() (Methode von sparse\_erw.Sparse)}

\begin{fulllineitems}
\phantomsection\label{\detokenize{index:sparse_erw.Sparse.lgs_lsg}}\pysiglinewithargsret{\sphinxbfcode{lgs\_lsg}}{\emph{r\_s=None}}{}~\begin{quote}

Loest das Gleichungssystem Ax=r\_s für eine vorgebene rechte Seite unter Ausnutzung der
Dreieckszerlegung.

Input:
\begin{quote}
\begin{description}
\item[{r\_s (numpy.ndarray):}] \leavevmode
rechte Seite b des zu loesenden Gleichungssystems A\textasciicircum{}(d)*x=b.

\end{description}
\end{quote}
\end{quote}
\begin{description}
\item[{Return:}] \leavevmode\begin{description}
\item[{(numpy.ndarray):}] \leavevmode
Loesungsvektor.

\end{description}

\end{description}

\end{fulllineitems}

\index{return\_mat\_d() (Methode von sparse\_erw.Sparse)}

\begin{fulllineitems}
\phantomsection\label{\detokenize{index:sparse_erw.Sparse.return_mat_d}}\pysiglinewithargsret{\sphinxbfcode{return\_mat\_d}}{}{}
Diese Methode gibt die Matrix A\textasciicircum{}(d) as sparse-Matrix zurueck.

Input: -
\begin{description}
\item[{Return:}] \leavevmode\begin{description}
\item[{(scipy.sparse.dok\_matrix-Objekt):}] \leavevmode
Die Matrix A\textasciicircum{}(d) als sparse-Matrix.

\end{description}

\end{description}

\end{fulllineitems}

\index{return\_mat\_d\_csc() (Methode von sparse\_erw.Sparse)}

\begin{fulllineitems}
\phantomsection\label{\detokenize{index:sparse_erw.Sparse.return_mat_d_csc}}\pysiglinewithargsret{\sphinxbfcode{return\_mat\_d\_csc}}{}{}
Gibt A\textasciicircum{}(d) als scipy.sparse.csc\_matrix-Objekt zurueck.

Input: -
\begin{description}
\item[{Return:}] \leavevmode\begin{description}
\item[{(scipy.sparse.csc\_matrix-Objekt)}] \leavevmode
Die Matrix A\textasciicircum{}(d)

\end{description}

\end{description}

\end{fulllineitems}

\index{return\_mat\_d\_inv() (Methode von sparse\_erw.Sparse)}

\begin{fulllineitems}
\phantomsection\label{\detokenize{index:sparse_erw.Sparse.return_mat_d_inv}}\pysiglinewithargsret{\sphinxbfcode{return\_mat\_d\_inv}}{}{}
Gibt die numerisch berechnete Inverse von A\textasciicircum{}(d) zurueck.

Input: -
\begin{description}
\item[{Return:}] \leavevmode\begin{description}
\item[{(scipy.sparse.dok\_matrix-Objekt):}] \leavevmode
Inverse von A\textasciicircum{}(d).

\end{description}

\end{description}

\end{fulllineitems}


\end{fulllineitems}



\renewcommand{\indexname}{Python-Modulindex}
\begin{sphinxtheindex}
\def\bigletter#1{{\Large\sffamily#1}\nopagebreak\vspace{1mm}}
\bigletter{s}
\item {\sphinxstyleindexentry{sparse\_erw}}\sphinxstyleindexpageref{index:\detokenize{module-sparse_erw}}
\end{sphinxtheindex}

\renewcommand{\indexname}{Stichwortverzeichnis}
\printindex
\end{document}