%% Generated by Sphinx.

\def\sphinxdocclass{report}
\documentclass[a4paper,10pt,ngerman, openright]{sphinxmanual}
\ifdefined\pdfpxdimen
   \let\sphinxpxdimen\pdfpxdimen\else\newdimen\sphinxpxdimen
\fi \sphinxpxdimen=.75bp\relax

\usepackage[utf8]{inputenc}
\ifdefined\DeclareUnicodeCharacter
 \ifdefined\DeclareUnicodeCharacterAsOptional
  \DeclareUnicodeCharacter{"00A0}{\nobreakspace}
  \DeclareUnicodeCharacter{"2500}{\sphinxunichar{2500}}
  \DeclareUnicodeCharacter{"2502}{\sphinxunichar{2502}}
  \DeclareUnicodeCharacter{"2514}{\sphinxunichar{2514}}
  \DeclareUnicodeCharacter{"251C}{\sphinxunichar{251C}}
  \DeclareUnicodeCharacter{"2572}{\textbackslash}
 \else
  \DeclareUnicodeCharacter{00A0}{\nobreakspace}
  \DeclareUnicodeCharacter{2500}{\sphinxunichar{2500}}
  \DeclareUnicodeCharacter{2502}{\sphinxunichar{2502}}
  \DeclareUnicodeCharacter{2514}{\sphinxunichar{2514}}
  \DeclareUnicodeCharacter{251C}{\sphinxunichar{251C}}
  \DeclareUnicodeCharacter{2572}{\textbackslash}
 \fi
\fi
\usepackage{cmap}
\usepackage[T1]{fontenc}
\usepackage{amsmath,amssymb,amstext}
\usepackage[ngerman]{babel}
\usepackage{times}
\usepackage[Sonny]{fncychap}
\usepackage[dontkeepoldnames]{sphinx}

\usepackage{geometry}

% Include hyperref last.
\usepackage{hyperref}
% Fix anchor placement for figures with captions.
\usepackage{hypcap}% it must be loaded after hyperref.
% Set up styles of URL: it should be placed after hyperref.
\urlstyle{same}
\addto\captionsngerman{\renewcommand{\contentsname}{Inhalt}}

\addto\captionsngerman{\renewcommand{\figurename}{Abb.}}
\addto\captionsngerman{\renewcommand{\tablename}{Tab.}}
\addto\captionsngerman{\renewcommand{\literalblockname}{Quellcode}}

\addto\captionsngerman{\renewcommand{\literalblockcontinuedname}{continued from previous page}}
\addto\captionsngerman{\renewcommand{\literalblockcontinuesname}{continues on next page}}

\addto\extrasngerman{\def\pageautorefname{Seite}}

\setcounter{tocdepth}{1}

\usepackage{nicefrac}


\title{Schnittstellendokumentation Serie 2 Documentation}
\date{18.11.2018}
\release{}
\author{Arsen Hnatiuk, Max Huneshagen}
\newcommand{\sphinxlogo}{\vbox{}}
\renewcommand{\releasename}{Release}
\makeindex

\begin{document}
\let\cleardoublepage\clearpage
\ifnum\catcode`\"=\active\shorthandoff{"}\fi
\maketitle
\sphinxtableofcontents
\phantomsection\label{\detokenize{index::doc}}


\chapter{Einleitung}
\let\cleardoublepage\clearpage
\chapter{Die Sparse-Klasse}
\label{\detokenize{index:welcome-to-schnittstellendokumentation-serie-2-s-documentation}}\label{\detokenize{index:module-sparse}}\label{\detokenize{index:die-sparse-klasse}}\index{sparse (Modul)}
sparse.py stellt die Klasse Sparse zur Verfuegung, mit der die Matrix A\textasciicircum{}(d) fuer d=1,2,3
bestimmt und analysiert werden kann.
\index{Sparse (Klasse in sparse)}

\begin{fulllineitems}
\phantomsection\label{\detokenize{index:sparse.Sparse}}\pysiglinewithargsret{\sphinxbfcode{class }\sphinxcode{sparse.}\sphinxbfcode{Sparse}}{\emph{dim}, \emph{dis}}{}
Diese Klasse erlaubt das Erstellen der Matrizen $A^{(d)}$ für $d\in \{ 1,2,3 \}$. Diese Matrizen werden
z.\,B. für die Berechnung der DGL $u'(x)=-f(x)$ verwendet. Es handelt sich bei diesen Matrizen
um sehr dünn besetzte Block-Band-Matrizen, was die Verwendung von sog. sparse-Matrizen
in der numerischen Umsetzung nahelegt.

Attribute:
\begin{quote}
\begin{description}
\item[{dim (int):}] \leavevmode
Raumdimension des zu untersuchenden Gebietes.

\item[{dis (numpy.ndarray aus floats):}] \leavevmode
Maß fuer die Diskretisierung des zu untersuchenden Gebietes.

\item[{matr (scipy.dok\_matrix-Objekt):}] \leavevmode
A\textasciicircum{}(d) mit Diskretisierung dis.

\end{description}
\end{quote}
\index{anz\_n\_abs() (Methode von sparse.Sparse)}

\begin{fulllineitems}
\phantomsection\label{\detokenize{index:sparse.Sparse.anz_n_abs}}\pysiglinewithargsret{\sphinxbfcode{anz\_n\_abs}}{}{}
Gibt die Anzahl von Eintraegen von A\textasciicircum{}(d) zurueck, die gleich 0 sind.

\begin{description}
\item[{Input:}] -\leavevmode
\item[{Return:}] \leavevmode\begin{description}
\item[{(int):}] \leavevmode
Anzahl von Nulleinträgen von $A^{(d)}$.

\end{description}

\end{description}

\end{fulllineitems}

\index{anz\_n\_rel() (Methode von sparse.Sparse)}

\begin{fulllineitems}
\phantomsection\label{\detokenize{index:sparse.Sparse.anz_n_rel}}\pysiglinewithargsret{\sphinxbfcode{anz\_n\_rel}}{}{}
Gibt die relative Anzahl von Eintraegen von A\textasciicircum{}(d) zurueck, die gleich 0 sind.

Input: -
\begin{description}
\item[{Return:}] \leavevmode\begin{description}
\item[{(int):}] \leavevmode
Relative Anzahl von Nulleintraegen von A\textasciicircum{}(d).

\end{description}

\end{description}

\end{fulllineitems}

\index{anz\_nn\_abs() (Methode von sparse.Sparse)}

\begin{fulllineitems}
\phantomsection\label{\detokenize{index:sparse.Sparse.anz_nn_abs}}\pysiglinewithargsret{\sphinxbfcode{anz\_nn\_abs}}{}{}
Gibt die Anzahl von Eintraegen von A\textasciicircum{}(d) zurueck, die ungleich 0 sind.

Input: -
\begin{description}
\item[{Return:}] \leavevmode\begin{description}
\item[{(int):}] \leavevmode
Anzahl von Nicht-Nulleintraegen von A\textasciicircum{}(d).

\end{description}

\end{description}

\end{fulllineitems}

\index{anz\_nn\_rel() (Methode von sparse.Sparse)}

\begin{fulllineitems}
\phantomsection\label{\detokenize{index:sparse.Sparse.anz_nn_rel}}\pysiglinewithargsret{\sphinxbfcode{anz\_nn\_rel}}{}{}
Gibt die relative Anzahl von Eintraegen von A\textasciicircum{}(d) zurueck, die gleich 0 sind.

Input: -
\begin{description}
\item[{Return:}] \leavevmode\begin{description}
\item[{(int):}] \leavevmode
Relative Anzahl von Nulleintraegen von A\textasciicircum{}(d).

\end{description}

\end{description}

\end{fulllineitems}

\index{constr\_mat\_l\_k() (Methode von sparse.Sparse)}

\begin{fulllineitems}
\phantomsection\label{\detokenize{index:sparse.Sparse.constr_mat_l_k}}\pysiglinewithargsret{\sphinxbfcode{constr\_mat\_l\_k}}{\emph{k}, \emph{dim}, \emph{dis}}{}
Konstruiert die Matrix A\_l(k) mit der gewuenschten Diskretisierung.

Input:
\begin{quote}
\begin{description}
\item[{k (float):}] \leavevmode
Bestimmt den Wert auf der Hauptdiagonalen der untersuchten Matrix (=2*k)

\item[{dim (int, moegliche Werte: 1, 2, 3):}] \leavevmode
Raumdimension des betrachteten Gebietes.

\item[{dis (int):}] \leavevmode
Diskretisierung des Gebietes.

\end{description}
\end{quote}
\begin{description}
\item[{Return:}] \leavevmode\begin{description}
\item[{(scipy.sparse.dok\_matrix-Objekt):}] \leavevmode
A\_l(k) mit der gewuenschten Diskretisierung.

\end{description}

\end{description}

\end{fulllineitems}

\index{return\_mat\_d() (Methode von sparse.Sparse)}

\begin{fulllineitems}
\phantomsection\label{\detokenize{index:sparse.Sparse.return_mat_d}}\pysiglinewithargsret{\sphinxbfcode{return\_mat\_d}}{}{}
Diese Methode gibt die Matrix A\textasciicircum{}(d) as sparse-Matrix zurueck.

Input: -
\begin{description}
\item[{Return:}] \leavevmode\begin{description}
\item[{(scipy.sparse.dok\_matrix-Objekt):}] \leavevmode
Die Matrix A\textasciicircum{}(d) als sparse-Matrix.

\end{description}

\end{description}

\end{fulllineitems}


\end{fulllineitems}



%\renewcommand{\indexname}{Python-Modulindex}
%\begin{sphinxtheindex}
%\def\bigletter#1{{\Large\sffamily#1}\nopagebreak\vspace{1mm}}
%\bigletter{s}
%\item {\sphinxstyleindexentry{sparse}}\sphinxstyleindexpageref{index:\detokenize{module-sparse}}
%\end{sphinxtheindex}

%\renewcommand{\indexname}{Stichwortverzeichnis}
%\printindex
\end{document}