%% Generated by Sphinx.

\def\sphinxdocclass{report}
\documentclass[a4paper,10pt,ngerman, oneside,openright]{sphinxmanual}
\ifdefined\pdfpxdimen
   \let\sphinxpxdimen\pdfpxdimen\else\newdimen\sphinxpxdimen
\fi \sphinxpxdimen=.75bp\relax

\usepackage[utf8]{inputenc}
\ifdefined\DeclareUnicodeCharacter
 \ifdefined\DeclareUnicodeCharacterAsOptional
  \DeclareUnicodeCharacter{"00A0}{\nobreakspace}
  \DeclareUnicodeCharacter{"2500}{\sphinxunichar{2500}}
  \DeclareUnicodeCharacter{"2502}{\sphinxunichar{2502}}
  \DeclareUnicodeCharacter{"2514}{\sphinxunichar{2514}}
  \DeclareUnicodeCharacter{"251C}{\sphinxunichar{251C}}
  \DeclareUnicodeCharacter{"2572}{\textbackslash}
 \else
  \DeclareUnicodeCharacter{00A0}{\nobreakspace}
  \DeclareUnicodeCharacter{2500}{\sphinxunichar{2500}}
  \DeclareUnicodeCharacter{2502}{\sphinxunichar{2502}}
  \DeclareUnicodeCharacter{2514}{\sphinxunichar{2514}}
  \DeclareUnicodeCharacter{251C}{\sphinxunichar{251C}}
  \DeclareUnicodeCharacter{2572}{\textbackslash}
 \fi
\fi
\usepackage{cmap}
\usepackage[T1]{fontenc}
\usepackage{amsmath,amssymb,amstext}
\usepackage[ngerman]{babel}
\usepackage{times}

%\usepackage{fancyhdr}
%\pagestyle{fancy}
%\fancyfoot{}
%\fancyfoot[C]{\thepage}
%\fancypagestyle{fancy}{%
%  \fancyfoot[C]{\thepage}%
%}

\usepackage[Sonny]{fncychap}
\usepackage[dontkeepoldnames]{sphinx}

\usepackage{geometry}

% Include hyperref last.
\usepackage{hyperref}
% Fix anchor placement for figures with captions.
\usepackage{hypcap}% it must be loaded after hyperref.
% Set up styles of URL: it should be placed after hyperref.
\urlstyle{same}
\addto\captionsngerman{\renewcommand{\contentsname}{Inhalt}}

\addto\captionsngerman{\renewcommand{\figurename}{Abb.}}
\addto\captionsngerman{\renewcommand{\tablename}{Tab.}}
\addto\captionsngerman{\renewcommand{\literalblockname}{Quellcode}}

\addto\captionsngerman{\renewcommand{\literalblockcontinuedname}{continued from previous page}}
\addto\captionsngerman{\renewcommand{\literalblockcontinuesname}{continues on next page}}

\addto\extrasngerman{\def\pageautorefname{Seite}}

\setcounter{tocdepth}{1}

\usepackage{nicefrac}
%\usepackage{fancyhdr}
%\fancyfoot[C]{\thepage}
%\pagestyle{fancy}


\title{Schnittstellendokumentation Serie~2 Teil~1}
\date{18.11.2018}
\release{}
\author{Arsen Hnatiuk, Max Huneshagen}
\newcommand{\sphinxlogo}{\vbox{}}
\renewcommand{\releasename}{}
\makeindex

\begin{document}
\let\cleardoublepage\clearpage
\ifnum\catcode`\"=\active\shorthandoff{"}\fi
\maketitle
\sphinxtableofcontents
\phantomsection\label{\detokenize{index::doc}}


\chapter{Einleitung}

Zur numerischen Lösung von Differenzialgleichungen der Form 
\begin{align}
u'(x)=-f(x)
\label{eq:dgl}
\end{align}
werden Matrizen benutzt, die die Überführung des Ausgangsproblems in ein Gleichungssystem
\begin{align}
Ax=b	
\label{eq:lgs}
\end{align}
mit $A\in\mathbb{R}^{m\times m}$, $x,b\in \mathbb{R}^m$ mit $m\in\mathbb{N}$ ermöglichen. Die Dimensionalität solcher Matrizen hängt von der Raumdimension $d$ und der Feinheit der Diskretisierung $n$ des betrachteten Gebietes ab. Zunächst definiert man für $d=1$:

\begin{align}
A^{(1)}:= 
\begin{pmatrix}
2 & -1 & 0 & \dots & 0\\
-1 & 2 & -1 & \dots & 0\\
\vdots & \ddots & \ddots & \ddots & \vdots\\
0 & 0 & -1 & 2 & -1\\
0 & 0 & \dots & -1 & 2
\end{pmatrix}\in \mathbb{R}^{(n-1)\times(n-1)}.
\label{eq:a_1_def}
\end{align}

Definiert man mit Hilfe der $(l\times l)$-Einheitsmatrix $I^l\in \mathbb{R}^{l\times l}$ die folgenden Matrizen:

\begin{align}
\mathcal{I}_d&:=I_{(n-1)^d}\in\mathbb{R}^{(n-1)^d\times(n-1)^d}\\
\mathcal{A}_1(k)&:=
\begin{pmatrix}
2k & -1 & 0 & \dots & 0\\
-1 & 2k & -1 & \dots & 0\\
\vdots & \ddots & \ddots & \ddots & \vdots \\
0 & 0 & -1 & 2k & -1\\
0 & 0 & \dots & -1 & 2k\\
\end{pmatrix} \in \mathbb{R}^{(n-1)\times(n-1)},
\end{align}
so lässt sich die folgende rekursive Matrixfolge definieren:

\begin{align}
\mathcal{A}_l(k):=
\begin{pmatrix}
\mathcal{A}_{l-1}(k) & -\mathcal{I}_{l-1} & 0 & 0 & \dots & 0\\
-\mathcal{I}_{l-1} & \mathcal{A}_{l-1}(k) & -\mathcal{I}_{l-1} & 0 & \dots & 0\\
0 & \ddots & \ddots & \ddots & \ddots & \vdots \\
\vdots & \ddots & \ddots & \ddots & \ddots & \vdots \\
0 & \dots & 0 & -\mathcal{I}_{l-1} & \mathcal{A}_{l-1}(k) & -\mathcal{I}_{l-1} \\
0 & \dots & 0 & 0 & -\mathcal{I}_{l-1} & \mathcal{A}_{l-1}(k)
\end{pmatrix} \in \mathbb{R}^{(n-1)^l\times(n-1)^l}.
\end{align}

Die Matrizen, die für die Lösung von~\eqref{eq:dgl}  bzw.~\eqref{eq:lgs} interessant sind, egeben sich für den Fall $l=k=d$:

\begin{align}
A^{(1)}&=\mathcal{A}_1(1)\text{~~~~~(in Übereinstimmung mit \eqref{eq:a_1_def})}\\
A^{(2)}&:=\mathcal{A}_2(2)\\
A^{(3)}&:=\mathcal{A}_3(3).
\end{align}

Diese Erstellung und Analyse dieser Matrizen wird in den hier beschriebenen Skripten behandelt.

\chapter{Die Sparse-Klasse}
\label{\detokenize{index:welcome-to-schnittstellendokumentation-serie-2-s-documentation}}\label{\detokenize{index:module-sparse}}\label{\detokenize{index:die-sparse-klasse}}\index{sparse (Modul)}
sparse.py stellt die Klasse \emph{sparse.Sparse} zur Verfügung, mit der die Matrix $A^{(d)}$ für $d\in\{1,2,3\}$
bestimmt und analysiert werden kann.
\index{Sparse (Klasse in sparse)}

\begin{fulllineitems}
\phantomsection\label{\detokenize{index:sparse.Sparse}}\pysiglinewithargsret{\sphinxbfcode{class }\sphinxcode{sparse.}\sphinxbfcode{Sparse}}{\emph{dim}, \emph{dis}}{}
Diese Klasse erlaubt das Erstellen der Matrizen $A^{(d)}$ für $d\in \{ 1,2,3 \}$. Diese Matrizen werden
z.\,B. für die Berechnung der DGL $u'(x)=-f(x)$ verwendet. Es handelt sich bei diesen Matrizen
um sehr dünn besetzte Block-Band-Matrizen, was die Verwendung von sog. sparse-Matrizen
in der numerischen Umsetzung nahelegt.

Attribute:
\begin{quote}
\begin{description}
\item[{dim (int):}] \leavevmode
Raumdimension des zu untersuchenden Gebietes.

\item[{dis (numpy.ndarray aus floats):}] \leavevmode
Maß für die Diskretisierung des zu untersuchenden Gebietes.

\item[{matr (scipy.dok\_matrix-Objekt):}] \leavevmode
$A^{(d)}$ mit angegebener Diskretisierung.

\end{description}
\end{quote}
\index{anz\_n\_abs() (Methode von sparse.Sparse)}

\begin{fulllineitems}
\phantomsection\label{\detokenize{index:sparse.Sparse.anz_n_abs}}\pysiglinewithargsret{\sphinxbfcode{anz\_n\_abs}}{}{}
Gibt die Anzahl von Einträgen von $A^{(d)}$ zurück, die gleich 0 sind.

\begin{description}
\item[{Input:}] -\leavevmode
\item[{Return:}] \leavevmode\begin{description}
\item[{(int):}] \leavevmode
Anzahl von Nulleinträgen von $A^{(d)}$.

\end{description}

\end{description}

\end{fulllineitems}

\index{anz\_n\_rel() (Methode von sparse.Sparse)}

\begin{fulllineitems}
\phantomsection\label{\detokenize{index:sparse.Sparse.anz_n_rel}}\pysiglinewithargsret{\sphinxbfcode{anz\_n\_rel}}{}{}
Gibt die relative Anzahl von Einträgen von $A^{(d)}$ zurück, die gleich 0 sind.

\begin{description}
\item[{Input:}] -\leavevmode
\item[{Return:}] \leavevmode\begin{description}
\item[{(int):}] \leavevmode
Relative Anzahl von Nulleinträgen von $A^{(d)}$.

\end{description}

\end{description}

\end{fulllineitems}

\index{anz\_nn\_abs() (Methode von sparse.Sparse)}

\clearpage
\begin{fulllineitems}
\phantomsection\label{\detokenize{index:sparse.Sparse.anz_nn_abs}}\pysiglinewithargsret{\sphinxbfcode{anz\_nn\_abs}}{}{}
Gibt die Anzahl von Einträgen von $A^{(d)}$ zurück, die ungleich 0 sind.

\begin{description}
\item[{Input:}] -\leavevmode
\item[{Return:}] \leavevmode\begin{description}
\item[{(int):}] \leavevmode
Anzahl von Nicht-Nulleinträgen von $A^{(d)}$.

\end{description}

\end{description}

\end{fulllineitems}

\index{anz\_nn\_rel() (Methode von sparse.Sparse)}

\begin{fulllineitems}
\phantomsection\label{\detokenize{index:sparse.Sparse.anz_nn_rel}}\pysiglinewithargsret{\sphinxbfcode{anz\_nn\_rel}}{}{}
Gibt die relative Anzahl von Einträgen von $A^{(d)}$ zurück, die ungleich 0 sind.

\begin{description}
\item[{Input:}] -\leavevmode
\item[{Return:}] \leavevmode\begin{description}
\item[{(int):}] \leavevmode
Relative Anzahl von Nicht-Nulleinträgen von $A^{(d)}$.

\end{description}

\end{description}

\end{fulllineitems}

\index{constr\_mat\_l\_k() (Methode von sparse.Sparse)}

\begin{fulllineitems}
\phantomsection\label{\detokenize{index:sparse.Sparse.constr_mat_l_k}}\pysiglinewithargsret{\sphinxbfcode{constr\_mat\_l\_k}}{\emph{k}, \emph{dim}, \emph{dis}}{}
Konstruiert die Matrix $A_l(k)$ mit der gewünschten Diskretisierung.

\begin{description}
\item[{Input:}]\leavevmode
\begin{quote}
\begin{description}
\item[{k (float):}] \leavevmode
Bestimmt den Wert auf der Hauptdiagonalen der untersuchten Matrix (=2*k)

\item[{dim (int, mögliche Werte: 1, 2, 3):}] \leavevmode
Raumdimension des betrachteten Gebietes.

\item[{dis (int):}] \leavevmode
Diskretisierung des Gebietes.
\end{description}

\end{quote}
\end{description}
\begin{description}
\item[{Return:}] \leavevmode\begin{description}
\item[{(scipy.sparse.dok\_matrix-Objekt):}] \leavevmode
$A_l(k)$ mit der gewünschten Diskretisierung.

\end{description}

\end{description}

\end{fulllineitems}

\index{return\_mat\_d() (Methode von sparse.Sparse)}

\begin{fulllineitems}
\phantomsection\label{\detokenize{index:sparse.Sparse.return_mat_d}}\pysiglinewithargsret{\sphinxbfcode{return\_mat\_d}}{}{}
Diese Methode gibt die Matrix $A^{(d)}$ as sparse-Matrix zurück.


\begin{description}
\item[{Input:}] \leavevmode
\item[{Return:}] \leavevmode\begin{description}
\item[{(scipy.sparse.dok\_matrix-Objekt):}] \leavevmode
Die Matrix $A^{(d)}$ als sparse-Matrix.

\end{description}

\end{description}

\end{fulllineitems}


\end{fulllineitems}

\chapter{hauptprogramm\_2\_1.py}

Das Hauptprogramm dient der Demonstration der Sparse-Klasse. Der Nutzer kann durch eine Konsolen-Eingabe die Diskretisierung und die Raumdimension wählen. woraufhin ein entsprechendes Sparse-Objekt erstellt wird. Anschließend wird (jeweils relative und absolute) Anzahl der (Nicht-)Nulleinträge der Matrix $A^{(d)}$ mit der gewählten Dimensionalität ausgegeben. Ist die Matrix nicht größer als eine $(20\times20)$-Matrix, so wird die Matrix darüber hinaus auf der Konsole ausgegeben. Dafür wird die \emph{scipy.sparse.dok\_matrix.todense}-Methode benutzt.

Was bei der Benutzung der Sparse-Klasse auffält, ist, dass die Befüllung von  \emph{scipy.sparse.dok\_matrix}-Objekten im Vergleich zu \emph{numpy.ndarray}-Objekten sehr lange dauert. Grund hierfür ist, dass bei jedem Schreibvorgang, bei dem ein Matrix-Eintrag von 0 auf einen anderen Wert geändert wird, die sog.~\glqq sparsity\grqq, also die Anzahl der Nulleinträge ebenfalls geändert werden muss. Die Umsetzung dessen im \emph{dictionary of keys} ist ein vergleichsweise aufwendiger Prozess und führt zu langen Laufzeiten. Dafür haben die hier verwendeten sparse-Matrizen den Vorteil, dass sie weniger Speicher als gewöhnliche Arrays oder Matrizen beanspruchen. Bei der Erstellung der Klasse wurde auf die Geschwindigkeit der Prozesse geachtet, indem unnötige explizite Aufrufe von Matrixelementen vermieden wurden. Zudem wurde die Rekursivität der Implementierung reduziert, indem der Fall \texttt{l==2} explizit behandelt wurde, anstatt die Methode \emph{constr\_mat\_l\_k} für \texttt{l-1} aufzurufen. Dennoch ergeben Tests mit dem Hauptprogramm, dass für $d=3$ und $n>40$ bereits eine relativ lange Zeit nötig ist, um die Matrizen zu erstellen.


%\renewcommand{\indexname}{Python-Modulindex}
%\begin{sphinxtheindex}
%\def\bigletter#1{{\Large\sffamily#1}\nopagebreak\vspace{1mm}}
%\bigletter{s}
%\item {\sphinxstyleindexentry{sparse}}\sphinxstyleindexpageref{index:\detokenize{module-sparse}}
%\end{sphinxtheindex}

%\renewcommand{\indexname}{Stichwortverzeichnis}
%\printindex
\end{document}
