%%
%% This is file `example_DarkConsole.tex',
%% generated with the docstrip utility.
%%
%% The original source files were:
%%
%% examples_kmbeamer.dtx  (with options: `DarkConsole')
%% Copyright (c) 2011-2013 Kazuki Maeda <kmaeda@users.sourceforge.jp>
%% 
%% Distributable under the MIT License:
%% http://www.opensource.org/licenses/mit-license.php
%% 

%%% もし pdfTeX や LuaTeX を使うなら dvipdfmx オプションを外す.
% \documentclass[dvipdfmx]{beamer}

% Modified by LianTze Lim to work with fontspec/xelatex
\documentclass{beamer}
\usepackage[ngerman]{babel}
\usepackage{mathspec}
\usepackage{xeCJK}
\setCJKmainfont{IPAPMincho}
\setCJKsansfont{IPAGothic}
\setCJKmonofont{IPAGothic}

% You can set fonts for Latin script here
\setmainfont{FreeSerif}
\setsansfont{FreeSans}
\setmonofont{Latin Modern Mono}

\usetheme{DarkConsole}

\usepackage{array}
\newcolumntype{P}[1]{>{\centering\arraybackslash}p{#1}}

%%%%%%%%%%%%%%%%%%%%%%%%%%%%%%%%%%%%%%%%%%%%%%%%%%%%%%%%%%%%%%%%%
% Neue Sachen::
%%%%%%%%%%%%%%%%%%%%%%%%%%%%%%%%%%%%%%%%%%%%%%%%%%%%%%%%%%%%%%%%%
%\addtobeamertemplate{frametitle}{
%   \let\insertframetitle\insertsectionhead}{}
%\addtobeamertemplate{frametitle}{
%   \let\insertframesubtitle\insertsubsectionhead}{}
%
%
%\makeatletter
%  \CheckCommand*\beamer@checkframetitle{\@ifnextchar\bgroup\beamer@inlineframetitle{}}
%  \renewcommand*\beamer@checkframetitle{\global\let\beamer@frametitle\relax\@ifnextchar\bgroup\beamer@inlineframetitle{}}
%\makeatother
\usepackage{graphicx}
\usepackage{array}
%%%%%%%%%%%%%%%%%%%%%%%%%%%%%%%%%%%%%%%%%%%%%%%%%%%%%%%%%%%%%%%%%
% Ende neue Sachen
%%%%%%%%%%%%%%%%%%%%%%%%%%%%%%%%%%%%%%%%%%%%%%%%%%%%%%%%%%%%%%%%%

%% \AtBeginDvi{\special{pdf:tounicode EUC-UCS2}} % EUC の場合
%% \AtBeginDvi{\special{pdf:tounicode 90ms-RKSJ-UCS2}} % SJIS の場合

%%% もし LuaTeX で日本語を出力するなら以下をコメントアウト.
%% \usefonttheme{luatexja}
%% \hypersetup{unicode}

%%% 日本語を使うなら以下を入れると定理環境中のフォントが立体になる.
%%% 欧文なら不要.
%%% LLT: Comment out this line if your presentation is in English or other European languages
\setbeamertemplate{theorems}[normal font]

\title{Projektpraktikum I}
\subtitle{Serie 4, Teil 1}
\author{Arsen Hnatiuk, Max Huneshagen}
\date{24. Januar 2019}

\begin{document}

\begin{frame}
  \maketitle
\end{frame}

\begin{frame}{Inhalt}
  \tableofcontents
  

  
 
\end{frame}
 \section{Methode der kleinsten Quadrate}  
\begin{frame}{Problemstellung [0]}
	\begin{itemize}
		\item Pegelstände an 3 verschiedenen Pegeln beobachtet:
		\begin{align}
		p, a_1, a_2 \in \mathbb{R}^m
		\end{align}
		\item Ziel: Pegelstand $p$ aus $a_1$ und $a_2$ vorhersagen
		\pause
		\item 2 Ansätze:
		\begin{itemize}
			\item Einfache lineare Regression:
			\begin{align}
			p =b_0+b_1a_1
			\label{eq:lin_reg}
			\end{align}
			\item Mehrfachregression:
			\begin{align}
			p =b_0+b_1a_1+b_2a_2
			\label{eq:mehr_reg}
			\end{align}
		\end{itemize}
	\end{itemize}
	
\end{frame}

\begin{frame}{Problemstellung [0]}
	\begin{itemize}
		\item \eqref{eq:lin_reg} und \eqref{eq:mehr_reg} führen auf die folgenden (überbestimmten) Gleichungssysteme:
		\begin{itemize}
			\item Lineare Regression:
			\begin{align}
			\begin{pmatrix}
			1 & \vrule \\
			\vdots & a_1\\
			1 & \vrule
			\end{pmatrix}
			\begin{pmatrix}
			b_0\\
			b_1
			\end{pmatrix}
			=
			\begin{pmatrix}
			\vrule\\
			p\\
			\vrule
			\end{pmatrix}
			\end{align}
			\item Mehrfachregression:
			\begin{align}
			\begin{pmatrix}
			1 & \vrule & \vrule\\
			\vdots & a_1&a_2\\
			1 & \vrule &\vrule
			\end{pmatrix}
			\begin{pmatrix}
			b_0\\
			b_1\\
			b_2
			\end{pmatrix}
			=
			\begin{pmatrix}
			\vrule\\
			p\\
			\vrule
			\end{pmatrix}
			\end{align}
		\pause
		\end{itemize}
		\item Lösung erfolgt über Methode der kleinsten Quadrate.
	\end{itemize}
\end{frame}


\begin{frame}{Herleitung}
  Sei $A\in \mathbb{R}^{n\times m}$ mit vollem Spaltenrang,  $b\in \mathbb{R}^n$ mit $n>m$.\\
  \begin{itemize}
  \item Ziel: \glqq Lösen\grqq ~des überbestimmten Gleichungssystems 
  \begin{align}
  Ax=b
  \end{align}
  durch Finden des Minimums
  \begin{align}
  \min\limits_{x\in\mathbb{R}^m}\|Ax-b\|_\infty.
  \end{align}\pause
\item Vorgehen:
\begin{itemize}
\item Finden der Q-R-Zerlegung:
\begin{align}
A=QR, \text{~~~}Q\in  O(n), R\in \mathbb{R}^{n\times m}\text{ obere Dreiecksmatrix}
\end{align}\pause
\item Mit $z:=Q^Tb=:
\begin{pmatrix}
z_1\\
\hline
z_2
\end{pmatrix}$ und $R=:\begin{pmatrix}
R_1\\
0
\end{pmatrix}$, wobei $z_1\in\mathbb{R}^{m}$ und $R_1\in\mathbb{R}^{m\times m}$ ergibt sich:
\begin{align}
R_1x=z_1\text{,~~~}  \min\limits_{x\in\mathbb{R}^m}\|Ax-b\|_\infty=\|z_2\|.
\end{align}

\end{itemize}
  \end{itemize}
  
\end{frame}

 \begin{frame}{Implementierung}
 	\begin{itemize}
  	\item Klasse \textit{KlQuad}:
  	\begin{itemize}
  		\item Q-R-Zerlegung der gegebenen Matrix und Berechnung des Residuums und der Kondition
  		\item Prüfung der Injektivität der Matrix (voller Spaltenrang)
  	\end{itemize}
  	\pause
  	\item Methode \textit{lese} Methode, die 
  	  	\begin{itemize}
  		\item Einlesen ausgewählter Zeilen und Spalten aus einer gegebenen Datei
  	\end{itemize}
  	\pause
  	\item \textit{main()}:
  	  	\begin{itemize}
  		\item Einlesen der Daten und grafische Darstellung
  	\end{itemize}
  	\end{itemize}
 \end{frame}

\section{Experimente}

\begin{frame}{Daten}
	\begin{table}\small
		\begin{tabular}{l || c c c c c c c c c c c c}
			$p$ & 172 & 309 & 302 & 283 & 443 & 298 & 319 & 419 & 361 & 267 & 337 & 230\\
			\hline
			$a_1$ & 93 & 193 & 187 & 174 & 291 & 184 & 205 & 260 &212 & 169 & 216 & 144\\
			\hline
			$a_2$ & 120 & 258 & 255 & 238 & 317 & 246 & 265 & 304 & 292 & 242 & 272 & 191\\
		\end{tabular}
		\caption{Ursprüngliche Daten}
	\end{table}
\end{frame}

\begin{frame}{Experimente}
	\begin{figure}
		\includegraphics[width=\textwidth]{Bilder/ungest+.png}
		\caption{Ungestörte Daten mit der dazugehörigen Lösung}
	\end{figure}
\end{frame}

\begin{frame}{Experimente}
	\begin{figure}
		\includegraphics[width=\textwidth]{Bilder/ungest+gerad.png}
		\caption{Nur die Hälfte der Daten berücksichtigt (gerade Indizes)}
	\end{figure}
 \end{frame}

\begin{frame}{Experimente}
	\begin{figure}
		\includegraphics[width=\textwidth]{Bilder/ungest+versch.png}
		\caption{Linear verschobene Daten}
	\end{figure}
\end{frame}

\begin{frame}{Experimente}
	\begin{figure}
		\includegraphics[width=\textwidth]{Bilder/ungest+rand.png}
		\caption{Daten mit einer zufälligen Störung}
	\end{figure}
\end{frame}

\begin{frame}{Vergleich von Residuen in den Experimenten}\small
	\begin{table}
		\begin{tabular}{p{2cm}|P{4.5cm}|P{4.5cm}}
			Experiment & Residuum (einfache Regression)&Residuum (Mehrfachregression)\\
			\hline
			Ohne Störung & 23.23&20.08\\
			\hline
			Hälfte der Daten& 17.18&15.56\\
			 \hline
			Linear verschobene Daten&23.23&20.08\\
			 \hline
			 Zufällig gestörte Daten&36.84&32.72\\
		\end{tabular}
		\caption{Berechnete Residuen}
	\end{table}
\end{frame}

\begin{frame}{Vergleich von Konditionen in den Experimenten}\small
	\begin{table}
		\begin{tabular}{p{2cm}|P{4cm}|P{4cm}}
			Experiment & Kondition von $A$ (einfache Regression)&Kondition von $A$ (Mehrfachregression)\\
			\hline
			Ohne Störung & 902.34 & 1934.43\\
			\hline
			Hälfte der Daten & 715.60 & 1442.70\\
			\hline
			Linear verschobene Daten & 976.16 & 2079.92\\
			\hline
			Zufällig gestörte Daten & 845.25 & 1941.25\\
		\end{tabular}
		\caption{Berechnete Konditionen}
	\end{table}
\end{frame}

\begin{frame}{Zusammenfassung}
	\begin{itemize}
		\item Genauigkeit  der mehrfachen Regression deutlich besser.
		\pause
		\item Störungen erhöhen Fehler, Verkleinerung der Datenmenge senkt ihn. 
		\pause
		\item
		Verschiebung von $a_1$ hat keine Auswirkung auf das Residiuum.
		\pause
		\item Kondition der Koeffizientenmatrix ist groß.
		\pause
		\item Mehrfachregression verdoppelt Kondition in etwa.
	\end{itemize}
\end{frame}

\end{document}
\endinput
%%
%% End of file `example_DarkConsole.tex'.
