%% Copyright 2018 H.\ Rabus
%
% This work may be distributed and/or modified under the
% conditions of the LaTeX Project Public License, either version 1.3
% of this license or (at your option) any later version.
% The latest version of this license is in
%   http://www.latex-project.org/lppl.txt
% and version 1.3 or later is part of all distributions of LaTeX
% version 2005/12/01 or later.
%
% This work has the LPPL maintenance status `author-maintained'.
%
% This work consists of the file texbsp.tex
%

\documentclass[smallheadings]{scrartcl}

%%% GENERAL PACKAGES %%%%%%%%%%%%%%%%%%%%%%%%%%%%%%%%%%%%%%%%%%%%%%%%%%%%%%%%%%
% inputenc allows the usage of non-ascii characters in the LaTeX source code
\usepackage[utf8]{inputenc}
\usepackage{graphicx} 
\usepackage{float}
%\graphicspath{ {/u/hnatiuka/Praktikum/PPI/} }



% title of the document
\title{Bericht zu Serie 5}
% optional subtitle
%\subtitle{Draft from~\today}
% information about the author
\author{%
  Arsen Hnatiuk,\\%
  Max Huneshagen 
}
\date{\today} 


%%% LANGUAGE %%%%%%%%%%%%%%%%%%%%%%%%%%%%%%%%%%%%%%%%%%%%%%%%%%%%%%%%%%%%%%%%%%
% babel provides hyphenation patterns and translations of keywords like 'table
% of contents'
\usepackage[ngerman]{babel}

%%% HYPERLINKS %%%%%%%%%%%%%%%%%%%%%%%%%%%%%%%%%%%%%%%%%%%%%%%%%%%%%%%%%%%%%%%%
% automatic generation of hyperlinks for references and URIs
\usepackage{hyperref}

%%% MATH %%%%%%%%%%%%%%%%%%%%%%%%%%%%%%%%%%%%%%%%%%%%%%%%%%%%%%%%%%%%%%%%%%%%%%
% amsmath provides commands for type-setting mathematical formulas
\usepackage{amsmath}
\numberwithin{equation}{section}
% amssymb provides additional symbols
\usepackage{amssymb}
% HINT
% Use http://detexify.kirelabs.org/classify.html to find unknown symbols!

%%% COLORS %%%%%%%%%%%%%%%%%%%%%%%%%%%%%%%%%%%%%%%%%%%%%%%%%%%%%%%%%%%%%%%%%%%%
% define own colors and use colored text
\usepackage[pdftex,svgnames,hyperref]{xcolor}

%%% Code Listings %%%%%%%%%%%%%%%%
% provides commands for including code (python, latex, ...)
\usepackage{listings}
\definecolor{keywords}{RGB}{255,0,90}
\definecolor{comments}{RGB}{0,0,113}
\definecolor{red}{RGB}{160,0,0}
\definecolor{green}{RGB}{0,150,0}
\lstset{language=Python, 
        basicstyle=\ttfamily\small, 
        keywordstyle=\color{keywords},
        commentstyle=\color{comments},
        stringstyle=\color{red},
        showstringspaces=false,
        identifierstyle=\color{green},
        }


\usepackage{paralist}
\usepackage{nicefrac}
% setting the font style for input und returns in description items
\newcommand{\initem}[2]{\item[\hspace{0.5em} {\normalfont\ttfamily{#1}} {\normalfont\itshape{(#2)}}]}
\newcommand{\outitem}[1]{\item[\hspace{0.5em} \normalfont\itshape{(#1)}]}
\newcommand{\bfpara}[1]{
	
	\noindent \textbf{#1:}\,}


\begin{document}

% generating the title page
\maketitle
% generating the table of contents (requires to run pdflatex twice!)
\tableofcontents
\bigskip

\hrule
\hrule

%%% BEGIN OF CONTENT %%%%%%%%%%%%%%%%%%%%%%%%%%%%%%%%%%%%%%%%%%%%%%%%%%%%%%%%%%

\section{Einleitung}

In dieser Serie soll das bereits in Serie~3 behandelte Problem 
\begin{align}
A^{(d)}\hat{u} = b
\label{eq:glc}
\end{align}
mit $A^{(d)}\in\mathbb{R}^{(n-1)^d\times(n-1)^d}$, $\hat{u},b\in\mathbb{R}^{n-1}$ und der Diskretisierung $n\in\mathbb{N}$ sowie $\hat{u},b \in \mathbb{R}^{(n-1)^d}$ erneut aufgegriffen werden. Dieses ist durch eine bestimmte Wahl von $A^{(d)}$ und $b$, die in Serie~2 erläutert wurde,  eine diskrete Formulierung des Laplace-Problems 
\begin{align}
\begin{split}
-\Delta u&=f(x)\text{ in }\Omega=(0, 1)^d, d\in\{1,2,3\}\\
u \vert _{\partial\Omega}&=u_D\text{ für ein gegebenes }u_D.
\end{split}
\label{eq:rwp}
\end{align}
Das exakte Lösen von \eqref{eq:glc} kann für feine Diskretisierungen einen sehr hohen Rechenaufwand verursachen. Im Laufe dessen wurde in Serie~3 die  $L$-$U$-Zerlegung von $A^{(d)}$ berechnet. Während $A^{(d)}$ eine Banddiagonalmatrix ist und sich demzufolge gut als \textit{sparse}-Matrix speichern lässt, ist dies für $L$ und $U$ im Allgemeinen nicht der Fall. Dieser Speicherplatzvorteil kann also bei der Verwendung der $L$-$U$-Zerlegung nicht länger ausgenutzt werden. Diese Beobachtungen motivieren die Verwendung von iterativen Verfahren wie der \emph{conjugate gradients}-Methode (\textbf{CG}). Bei dieser handelt es sich eigentlich um ein exaktes Verfahren, mit dem sich ein Gleichungssystem der Form
\begin{align}
Bx=d
\end{align}
mit $B \in \mathbb{R}^{m\times m}$ s.\,p.\,d. und $x,d\in \mathbb{R}^m$ in $m$ Iterationsschritten exakt lösen lässt. In der Praxis kann das Verfahren aber meist nach viel weniger Schritten abgebrochen werden, da die Zwischenlösung schon nahe genug (insbesondere im Vergleich zum Diskretisierungsfehler) an der exakten Lösung liegt.

Die \textbf{CG}-Methode soll im Folgenden verwendet werden, um \eqref{eq:rwp} für ein vorgegebenes $u$ bzw. \eqref{eq:glc} für eine vorgegebene rechte Seite näherungsweise zu lösen. Die so erhaltene approximative Lösung wird mit der Lösung mit dem bereits in Serie~3 verwendeten Verfahren hinsichtlich des Konvergenzverhaltens verglichen. 

\section{Theorie}


Sei $B \in \mathbb{R}^{m\times m}$ s.\,p.\,d. und $x,d\in \mathbb{R}^m$. Sei $x_0\in \mathbb{R}^m$ beliebig. Dann berechnet sich eine Lösung von $Bx=d$ wie folgt \cite{wiki:cg}:

%TODO Hier eventuell Einzug?

Zunächst sei 
\begin{align}
r_0&:=d-Bx_0\\
d_0&:=r_0.
\end{align}
Berechne nun für $k=0,1,2,\dots$:
\begin{align}
z&=Bd_k,\\
\alpha&=\frac{r_k^Tr_k}{d_k^Tz},\\
x_{k+1}&=x_k+\alpha_kd_k,\\
r_{k+1}&=r_k-\alpha_kz,\\
\beta_k&=\frac{r^T_{k+1}r_{k+1}}{r^T_{k}r_{k}},\\
d_{k+1}&=r_{k+1}+\beta_kd_k,
\end{align}
bis das Residuum $\|r_{k+1}\|_\infty$ kleiner als ein zuvor gegebenes $\epsilon$ wird.
Zu prüfen ist nun noch, ob die Matrizen $A^{(d)}$ die o.\,g. Bedingungen erfüllen und symmetrisch und positiv semidefinit sind. Symmetrie folgt direkt aus der Definition der Matrizen, die positive Definitheit folgt daraus, dass die $A^{(d)}$ streng diagonaldominante Matrizen sind, also dass gilt:
\begin{align}
|(A^{(d)})_{ii}|> \sum\limits_{\substack{j=1 \\j\neq i}}|(A^{(d)})_{ij}|.
\end{align}
Dies ist gegeben, da die Diagonaleinträge von $A^{(d)}$ den Betrag $2d$ (s.~Serie~2) haben und sich in jeder Zeile höchstens $d$ Einträge mit dem Betrag $1$ befinden, während die restlichen Beiträge $0$ sind. Benutzt man nun noch, dass alle Diagonaleinträge von $A^{(d)}$ stets positiv sind, so folgt daraus, dass diese Metrizen positiv definit sind \cite{skript:nla}. Demzufolge lässt sich die CG-Methode auf diese Matrizen anwenden. 






\section{Zusammenfassung}


\begin{thebibliography}{9}
\bibitem{wiki:cg} Kein Autor, Aufgerufen am \today, \textit{C-G-Verfahren}. 
\url{https://de.wikipedia.org/wiki/CG-Verfahren}
\bibitem{skript:nla} Axel~Kröner: Vorlesungsskript \textit{Numerische Lineare Algebra}, WS 2018/19. 
\url{https://moodle.hu-berlin.de/pluginfile.php/2445740/mod_resource/content/25/Skriptum_NLA.pdf} (passwortgeschützt)
\end{thebibliography}


%%% END OF DOCUMENT %%%%%%%%%%%%%%%%%%%%%%%%%%%%%%%%%%%%%%%%%%%%%%%%%%%%%%%%%%%
\end{document}
