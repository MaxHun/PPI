%% Generated by Sphinx.
\def\sphinxdocclass{report}
\documentclass[letterpaper,10pt,ngerman]{sphinxmanual}
\ifdefined\pdfpxdimen
   \let\sphinxpxdimen\pdfpxdimen\else\newdimen\sphinxpxdimen
\fi \sphinxpxdimen=.75bp\relax

\PassOptionsToPackage{warn}{textcomp}
\usepackage[utf8]{inputenc}
\ifdefined\DeclareUnicodeCharacter
 \ifdefined\DeclareUnicodeCharacterAsOptional
  \DeclareUnicodeCharacter{"00A0}{\nobreakspace}
  \DeclareUnicodeCharacter{"2500}{\sphinxunichar{2500}}
  \DeclareUnicodeCharacter{"2502}{\sphinxunichar{2502}}
  \DeclareUnicodeCharacter{"2514}{\sphinxunichar{2514}}
  \DeclareUnicodeCharacter{"251C}{\sphinxunichar{251C}}
  \DeclareUnicodeCharacter{"2572}{\textbackslash}
 \else
  \DeclareUnicodeCharacter{00A0}{\nobreakspace}
  \DeclareUnicodeCharacter{2500}{\sphinxunichar{2500}}
  \DeclareUnicodeCharacter{2502}{\sphinxunichar{2502}}
  \DeclareUnicodeCharacter{2514}{\sphinxunichar{2514}}
  \DeclareUnicodeCharacter{251C}{\sphinxunichar{251C}}
  \DeclareUnicodeCharacter{2572}{\textbackslash}
 \fi
\fi
\usepackage{cmap}
\usepackage[T1]{fontenc}
\usepackage{amsmath,amssymb,amstext}
\usepackage{babel}
\usepackage{times}
\usepackage[Sonny]{fncychap}
\ChNameVar{\Large\normalfont\sffamily}
\ChTitleVar{\Large\normalfont\sffamily}
\usepackage{sphinx}

\usepackage{geometry}

% Include hyperref last.
\usepackage{hyperref}
% Fix anchor placement for figures with captions.
\usepackage{hypcap}% it must be loaded after hyperref.
% Set up styles of URL: it should be placed after hyperref.
\urlstyle{same}
\addto\captionsngerman{\renewcommand{\contentsname}{Contents:}}

\addto\captionsngerman{\renewcommand{\figurename}{Abb.}}
\addto\captionsngerman{\renewcommand{\tablename}{Tab.}}
\addto\captionsngerman{\renewcommand{\literalblockname}{Quellcode}}

\addto\captionsngerman{\renewcommand{\literalblockcontinuedname}{Fortsetzung der vorherigen Seite}}
\addto\captionsngerman{\renewcommand{\literalblockcontinuesname}{Fortsetzung auf der nächsten Seite}}

\addto\extrasngerman{\def\pageautorefname{Seite}}

\setcounter{tocdepth}{1}



\title{Projektpraktikum 1, Serie 5 Documentation}
\date{31.01.2019}
\release{}
\author{Arsen Hnatiuk, Max Huneshagen}
\newcommand{\sphinxlogo}{\vbox{}}
\renewcommand{\releasename}{}
\makeindex

\begin{document}
\ifdefined\shorthandoff
  \ifnum\catcode`\=\string=\active\shorthandoff{=}\fi
  \ifnum\catcode`\"=\active\shorthandoff{"}\fi
\fi
\maketitle
\sphinxtableofcontents
\phantomsection\label{\detokenize{index::doc}}



\chapter{Die Sparse-Klasse}
\label{\detokenize{index:module-sparse_erw}}\label{\detokenize{index:die-sparse-klasse}}\index{sparse\_erw (Modul)}
sparse.py stellt die Klasse Sparse zur Verfuegung, mit der die Matrix A\textasciicircum{}(d) fuer d=1,2,3
bestimmt und analysiert werden kann.
\index{Sparse (Klasse in sparse\_erw)}

\begin{fulllineitems}
\phantomsection\label{\detokenize{index:sparse_erw.Sparse}}\pysiglinewithargsret{\sphinxbfcode{\sphinxupquote{class }}\sphinxcode{\sphinxupquote{sparse\_erw.}}\sphinxbfcode{\sphinxupquote{Sparse}}}{\emph{dim}, \emph{dis}, \emph{r\_s=None}, \emph{ex\_lsg=None}}{}
Diese Klasse erlaubt das Erstellen der Matrizen A\textasciicircum{}(d) fuer d in {[}1,2,3{]}. Diese Matrizen werden
z. B. fuer die Berechnung der DGL u‘‚(x)=-f(x) verwendet. Es handelt sich bei diesen Matrizen
um sehr duenn besetzte Block-Band-Matrizen, was die Verwendung von sog. sparse-Matrizen
in der numerischen Umsetzung nahelegt.

Attribute:
\begin{quote}
\begin{description}
\item[{dim (int):}] \leavevmode
Raumdimension des zu untersuchenden Gebietes.

\item[{dis (numpy.ndarray aus floats):}] \leavevmode
Mass fuer die Diskretisierung des zu untersuchenden Gebietes.

\item[{matr (scipy.dok\_matrix-Objekt):}] \leavevmode
A\textasciicircum{}(d) mit Diskretisierung dis.

\end{description}
\end{quote}
\index{anz\_n\_abs() (Methode von sparse\_erw.Sparse)}

\begin{fulllineitems}
\phantomsection\label{\detokenize{index:sparse_erw.Sparse.anz_n_abs}}\pysiglinewithargsret{\sphinxbfcode{\sphinxupquote{anz\_n\_abs}}}{}{}
Gibt die Anzahl von Eintraegen von A\textasciicircum{}(d) zurueck, die gleich 0 sind.

Input: -
\begin{description}
\item[{Return:}] \leavevmode\begin{description}
\item[{(int):}] \leavevmode
Anzahl von Nulleintraegen von A\textasciicircum{}(d).

\end{description}

\end{description}

\end{fulllineitems}

\index{anz\_n\_lu\_abs() (Methode von sparse\_erw.Sparse)}

\begin{fulllineitems}
\phantomsection\label{\detokenize{index:sparse_erw.Sparse.anz_n_lu_abs}}\pysiglinewithargsret{\sphinxbfcode{\sphinxupquote{anz\_n\_lu\_abs}}}{}{}
Gibt die Anzahl von Eintraegen von L bzw. U zurueck, die gleich 0 sind.

Input: -
\begin{description}
\item[{Return:}] \leavevmode\begin{description}
\item[{(int-Tupel):}] \leavevmode
Anzahl von Nulleintraegen von L und U.

\end{description}

\end{description}

\end{fulllineitems}

\index{anz\_n\_lu\_rel() (Methode von sparse\_erw.Sparse)}

\begin{fulllineitems}
\phantomsection\label{\detokenize{index:sparse_erw.Sparse.anz_n_lu_rel}}\pysiglinewithargsret{\sphinxbfcode{\sphinxupquote{anz\_n\_lu\_rel}}}{}{}
Gibt die relative Anzahl von Eintraegen von L bzw. U zurueck, die gleich 0 sind.

Input: -
\begin{description}
\item[{Return:}] \leavevmode\begin{description}
\item[{(int-Tupel):}] \leavevmode
Relative Anzahl von Nulleintraegen von L und U.

\end{description}

\end{description}

\end{fulllineitems}

\index{anz\_n\_rel() (Methode von sparse\_erw.Sparse)}

\begin{fulllineitems}
\phantomsection\label{\detokenize{index:sparse_erw.Sparse.anz_n_rel}}\pysiglinewithargsret{\sphinxbfcode{\sphinxupquote{anz\_n\_rel}}}{}{}
Gibt die relative Anzahl von Eintraegen von A\textasciicircum{}(d) zurueck, die gleich 0 sind.

Input: -
\begin{description}
\item[{Return:}] \leavevmode\begin{description}
\item[{(int):}] \leavevmode
Relative Anzahl von Nulleintraegen von A\textasciicircum{}(d).

\end{description}

\end{description}

\end{fulllineitems}

\index{anz\_nn\_abs() (Methode von sparse\_erw.Sparse)}

\begin{fulllineitems}
\phantomsection\label{\detokenize{index:sparse_erw.Sparse.anz_nn_abs}}\pysiglinewithargsret{\sphinxbfcode{\sphinxupquote{anz\_nn\_abs}}}{}{}
Gibt die Anzahl von Eintraegen von A\textasciicircum{}(d) zurueck, die ungleich 0 sind.

Input: -
\begin{description}
\item[{Return:}] \leavevmode\begin{description}
\item[{(int):}] \leavevmode
Anzahl von Nicht-Nulleintraegen von A\textasciicircum{}(d).

\end{description}

\end{description}

\end{fulllineitems}

\index{anz\_nn\_lu\_abs() (Methode von sparse\_erw.Sparse)}

\begin{fulllineitems}
\phantomsection\label{\detokenize{index:sparse_erw.Sparse.anz_nn_lu_abs}}\pysiglinewithargsret{\sphinxbfcode{\sphinxupquote{anz\_nn\_lu\_abs}}}{}{}
Gibt die Anzahl von Eintraegen von L bzw. U zurueck, die ungleich 0 sind.

Input: -
\begin{description}
\item[{Return:}] \leavevmode\begin{description}
\item[{(int-Tupel):}] \leavevmode
Anzahl von Nicht-Nulleintraegen von L und U.

\end{description}

\end{description}

\end{fulllineitems}

\index{anz\_nn\_lu\_rel() (Methode von sparse\_erw.Sparse)}

\begin{fulllineitems}
\phantomsection\label{\detokenize{index:sparse_erw.Sparse.anz_nn_lu_rel}}\pysiglinewithargsret{\sphinxbfcode{\sphinxupquote{anz\_nn\_lu\_rel}}}{}{}
Gibt die relative Anzahl von Eintraegen von L bzw. U zurueck, die ungleich 0 sind.

Input: -
\begin{description}
\item[{Return:}] \leavevmode\begin{description}
\item[{(int-Tupel):}] \leavevmode
Relative Anzahl von Nulleintraegen von L und U.

\end{description}

\end{description}

\end{fulllineitems}

\index{anz\_nn\_rel() (Methode von sparse\_erw.Sparse)}

\begin{fulllineitems}
\phantomsection\label{\detokenize{index:sparse_erw.Sparse.anz_nn_rel}}\pysiglinewithargsret{\sphinxbfcode{\sphinxupquote{anz\_nn\_rel}}}{}{}
Gibt die relative Anzahl von Eintraegen von A\textasciicircum{}(d) zurueck, die ungleich 0 sind.

Input: -
\begin{description}
\item[{Return:}] \leavevmode\begin{description}
\item[{(int):}] \leavevmode
Relative Anzahl von Nulleintraegen von A\textasciicircum{}(d).

\end{description}

\end{description}

\end{fulllineitems}

\index{cg\_meth() (Methode von sparse\_erw.Sparse)}

\begin{fulllineitems}
\phantomsection\label{\detokenize{index:sparse_erw.Sparse.cg_meth}}\pysiglinewithargsret{\sphinxbfcode{\sphinxupquote{cg\_meth}}}{\emph{los1}, \emph{vekb}, \emph{eps}}{}
Diese Methode approximiert die Loesung eines Gleichungssystems mithilfe des CG-Verfahrens,
bis auf eine vogegebene Genauigkeit.

Input:
\begin{quote}
\begin{description}
\item[{los0 (numpy.ndarray aus floats):}] \leavevmode
Die erste Loesung, die in der Iteration benutzt wird

\item[{vekb (numpy.ndarray aus floats):}] \leavevmode
Der b Vektor des zu loesendes Gleichungssystems

\item[{eps (float):}] \leavevmode
Die Genauigkeit, bei welcher die Iteration abgebrochen wird

\end{description}
\end{quote}

Return:
\begin{quote}
\begin{description}
\item[{los (Liste aus numpy.ndarray aus floats):}] \leavevmode
Liste mit den berechneten Loesungen nach jedem Iterationsschitt

\end{description}
\end{quote}

\end{fulllineitems}

\index{constr\_mat\_l\_k() (Methode von sparse\_erw.Sparse)}

\begin{fulllineitems}
\phantomsection\label{\detokenize{index:sparse_erw.Sparse.constr_mat_l_k}}\pysiglinewithargsret{\sphinxbfcode{\sphinxupquote{constr\_mat\_l\_k}}}{\emph{k}, \emph{dim}, \emph{dis}}{}
Konstruiert die Matrix A\_l(k) mit der gewuenschten Diskretisierung.

Input:
\begin{quote}
\begin{description}
\item[{k (float):}] \leavevmode
Bestimmt den Wert auf der Hauptdiagonalen der untersuchten Matrix (=2*k)

\item[{dim (int, moegliche Werte: 1, 2, 3):}] \leavevmode
Raumdimension des betrachteten Gebietes.

\item[{dis (int):}] \leavevmode
Diskretisierung des Gebietes.

\end{description}
\end{quote}
\begin{description}
\item[{Return:}] \leavevmode\begin{description}
\item[{(scipy.sparse.dok\_matrix-Objekt):}] \leavevmode
A\_l(k) mit der gewuenschten Diskretisierung.

\end{description}

\end{description}

\end{fulllineitems}

\index{kond\_a\_d\_zs() (Methode von sparse\_erw.Sparse)}

\begin{fulllineitems}
\phantomsection\label{\detokenize{index:sparse_erw.Sparse.kond_a_d_zs}}\pysiglinewithargsret{\sphinxbfcode{\sphinxupquote{kond\_a\_d\_zs}}}{}{}
Gibt die Kondition der Matrix A\textasciicircum{}(d) bezueglich der Zeilensummennorm zurueck.

Input: -
\begin{description}
\item[{Return:}] \leavevmode\begin{description}
\item[{(float):}] \leavevmode
Kondition von A\textasciicircum{}(d) bzgl. Zeilennorm.

\end{description}

\end{description}

\end{fulllineitems}

\index{l\_u\_zerl() (Methode von sparse\_erw.Sparse)}

\begin{fulllineitems}
\phantomsection\label{\detokenize{index:sparse_erw.Sparse.l_u_zerl}}\pysiglinewithargsret{\sphinxbfcode{\sphinxupquote{l\_u\_zerl}}}{}{}
Errechnet die L-U-Zerlegung von A\textasciicircum{}(d)=P\_r\textasciicircum{}T*L*U*P\_c\textasciicircum{}T

Input: -
\begin{description}
\item[{Return:}] \leavevmode\begin{description}
\item[{(list):}] \leavevmode
Nullter Eintrag: das mit der Zerlegung korrespondierende scipy.SuperLU-Objekt.
Erster Eintrag: Tripel aus den Matrizen P\_r, P\_c, L, U.

\end{description}

\end{description}

\end{fulllineitems}

\index{lgs\_lsg() (Methode von sparse\_erw.Sparse)}

\begin{fulllineitems}
\phantomsection\label{\detokenize{index:sparse_erw.Sparse.lgs_lsg}}\pysiglinewithargsret{\sphinxbfcode{\sphinxupquote{lgs\_lsg}}}{\emph{r\_s=None}}{}~\begin{quote}

Loest das Gleichungssystem Ax=r\_s fuer eine vorgebene rechte Seite unter Ausnutzung der
Dreieckszerlegung.

Input:
\begin{quote}
\begin{description}
\item[{r\_s (numpy.ndarray):}] \leavevmode
rechte Seite b des zu loesenden Gleichungssystems A\textasciicircum{}(d)*x=b.

\end{description}
\end{quote}
\end{quote}
\begin{description}
\item[{Return:}] \leavevmode\begin{description}
\item[{(numpy.ndarray):}] \leavevmode
Loesungsvektor.

\end{description}

\end{description}

\end{fulllineitems}

\index{return\_mat\_d() (Methode von sparse\_erw.Sparse)}

\begin{fulllineitems}
\phantomsection\label{\detokenize{index:sparse_erw.Sparse.return_mat_d}}\pysiglinewithargsret{\sphinxbfcode{\sphinxupquote{return\_mat\_d}}}{}{}
Diese Methode gibt die Matrix A\textasciicircum{}(d) as sparse-Matrix zurueck.

Input: -
\begin{description}
\item[{Return:}] \leavevmode\begin{description}
\item[{(scipy.sparse.dok\_matrix-Objekt):}] \leavevmode
Die Matrix A\textasciicircum{}(d) als sparse-Matrix.

\end{description}

\end{description}

\end{fulllineitems}

\index{return\_mat\_d\_csc() (Methode von sparse\_erw.Sparse)}

\begin{fulllineitems}
\phantomsection\label{\detokenize{index:sparse_erw.Sparse.return_mat_d_csc}}\pysiglinewithargsret{\sphinxbfcode{\sphinxupquote{return\_mat\_d\_csc}}}{}{}
Gibt A\textasciicircum{}(d) als scipy.sparse.csc\_matrix-Objekt zurueck.

Input: -
\begin{description}
\item[{Return:}] \leavevmode\begin{description}
\item[{(scipy.sparse.csc\_matrix-Objekt)}] \leavevmode
Die Matrix A\textasciicircum{}(d)

\end{description}

\end{description}

\end{fulllineitems}

\index{return\_mat\_d\_inv() (Methode von sparse\_erw.Sparse)}

\begin{fulllineitems}
\phantomsection\label{\detokenize{index:sparse_erw.Sparse.return_mat_d_inv}}\pysiglinewithargsret{\sphinxbfcode{\sphinxupquote{return\_mat\_d\_inv}}}{}{}
Gibt die numerisch berechnete Inverse von A\textasciicircum{}(d) zurueck.

Input: -
\begin{description}
\item[{Return:}] \leavevmode\begin{description}
\item[{(scipy.sparse.dok\_matrix-Objekt):}] \leavevmode
Inverse von A\textasciicircum{}(d).

\end{description}

\end{description}

\end{fulllineitems}


\end{fulllineitems}



\chapter{Das Skript aufg\_5-2.py}
\label{\detokenize{index:module-aufg_5_2}}\label{\detokenize{index:das-skript-aufg-5-2-py}}\index{aufg\_5\_2 (Modul)}
Dieses Modul loest die Aufgabe 5.2, indem es die Differentialgleichung loest und
das Verhalten der Loesungen grafisch aufstellt.
Autoren Arsen Hnatiuk und Max Huneshagen
\index{fntn1() (im Modul aufg\_5\_2)}

\begin{fulllineitems}
\phantomsection\label{\detokenize{index:aufg_5_2.fntn1}}\pysiglinewithargsret{\sphinxcode{\sphinxupquote{aufg\_5\_2.}}\sphinxbfcode{\sphinxupquote{fntn1}}}{\emph{wert}}{}
Beispielfunktion der Dimension 1
Input:
\begin{quote}
\begin{description}
\item[{wert (float):}] \leavevmode
Werte, auf dem die Funktion ausgewertet wird.

\end{description}
\end{quote}
\begin{description}
\item[{Return:}] \leavevmode\begin{description}
\item[{(float):}] \leavevmode
Wert der Funktion

\end{description}

\end{description}

\end{fulllineitems}

\index{fntn2() (im Modul aufg\_5\_2)}

\begin{fulllineitems}
\phantomsection\label{\detokenize{index:aufg_5_2.fntn2}}\pysiglinewithargsret{\sphinxcode{\sphinxupquote{aufg\_5\_2.}}\sphinxbfcode{\sphinxupquote{fntn2}}}{\emph{wert}}{}
Beispielfunktion der Dimension 2
Input:
\begin{quote}
\begin{description}
\item[{wert (array von float):}] \leavevmode
Werte, auf dem die Funktion ausgewertet wird.

\end{description}
\end{quote}
\begin{description}
\item[{Return:}] \leavevmode\begin{description}
\item[{(float):}] \leavevmode
Wert der Funktion

\end{description}

\end{description}

\end{fulllineitems}

\index{fntn3() (im Modul aufg\_5\_2)}

\begin{fulllineitems}
\phantomsection\label{\detokenize{index:aufg_5_2.fntn3}}\pysiglinewithargsret{\sphinxcode{\sphinxupquote{aufg\_5\_2.}}\sphinxbfcode{\sphinxupquote{fntn3}}}{\emph{wert}}{}
Beispielfunktion der Dimension 3
Input:
\begin{quote}
\begin{description}
\item[{wert (array von float):}] \leavevmode
Werte, auf dem die Funktion ausgewertet wird.

\end{description}
\end{quote}
\begin{description}
\item[{Return:}] \leavevmode\begin{description}
\item[{(float):}] \leavevmode
Wert der Funktion

\end{description}

\end{description}

\end{fulllineitems}

\index{gitter() (im Modul aufg\_5\_2)}

\begin{fulllineitems}
\phantomsection\label{\detokenize{index:aufg_5_2.gitter}}\pysiglinewithargsret{\sphinxcode{\sphinxupquote{aufg\_5\_2.}}\sphinxbfcode{\sphinxupquote{gitter}}}{\emph{numb}, \emph{dims}}{}
Funktion, die fuer eine gegebene Dimension und Schritweite eine Menge von Punkten erzeugt, auf
der die Differentialgleichung zu loesen ist.
Input:
\begin{quote}
\begin{description}
\item[{numb (int):}] \leavevmode
Feinheit der Diskretisierung

\item[{dims (int):}] \leavevmode
Dimension der Diskretisierung

\end{description}
\end{quote}
\begin{description}
\item[{Return:}] \leavevmode\begin{description}
\item[{arra (ndarray):}] \leavevmode
Array mit den Punkten auf dem Diskretisierungsgitter

\end{description}

\end{description}

\end{fulllineitems}

\index{loesg() (im Modul aufg\_5\_2)}

\begin{fulllineitems}
\phantomsection\label{\detokenize{index:aufg_5_2.loesg}}\pysiglinewithargsret{\sphinxcode{\sphinxupquote{aufg\_5\_2.}}\sphinxbfcode{\sphinxupquote{loesg}}}{\emph{dims}, \emph{numb}, \emph{fkt}, \emph{ulsg}}{}
Diese Methode dient zur Loesung der Differentialgleichung und zum Vergleichen der exakten
und approxmierten Loesungen.
Input:
\begin{quote}
\begin{description}
\item[{dims (int):}] \leavevmode
Dimension der Diskretisierung

\item[{numb (int):}] \leavevmode
Feinheit der DiskretisierungS

\item[{fkt (Funktion):}] \leavevmode
Die gegebene Funktion f aus der Aufgabestellung

\item[{ulsg (Funktion):}] \leavevmode
Die exakte Loesung der Differentialgleichung

\end{description}
\end{quote}
\begin{description}
\item[{Return:}] \leavevmode\begin{description}
\item[{(float):}] \leavevmode
Der absolute Fehler in der approximierten Loesung der Differentialgleichung

\end{description}

\end{description}

\end{fulllineitems}

\index{main() (im Modul aufg\_5\_2)}

\begin{fulllineitems}
\phantomsection\label{\detokenize{index:aufg_5_2.main}}\pysiglinewithargsret{\sphinxcode{\sphinxupquote{aufg\_5\_2.}}\sphinxbfcode{\sphinxupquote{main}}}{}{}
In dieser Funktion werden alle Loesungen der Aufgabe dem Nutzer ausgegeben

\end{fulllineitems}

\index{modo() (im Modul aufg\_5\_2)}

\begin{fulllineitems}
\phantomsection\label{\detokenize{index:aufg_5_2.modo}}\pysiglinewithargsret{\sphinxcode{\sphinxupquote{aufg\_5\_2.}}\sphinxbfcode{\sphinxupquote{modo}}}{\emph{m\_nb}, \emph{numb}}{}
Funktion, die die modofizierte Modolo Methode aus dem Bericht zu Serie 2 implementiert
Input:
\begin{quote}
\begin{description}
\item[{m\_nb (int):}] \leavevmode
Zahl, fuer die wir den Wert der modifizierten Modolo Methode berechnen wollen

\item[{numb (int):}] \leavevmode
Feinheit der Diskretisierung

\end{description}
\end{quote}
\begin{description}
\item[{Return:}] \leavevmode\begin{description}
\item[{(int):}] \leavevmode
Ausgabe der modifizierten Modolo Methode

\end{description}

\end{description}

\end{fulllineitems}

\index{ulsg1() (im Modul aufg\_5\_2)}

\begin{fulllineitems}
\phantomsection\label{\detokenize{index:aufg_5_2.ulsg1}}\pysiglinewithargsret{\sphinxcode{\sphinxupquote{aufg\_5\_2.}}\sphinxbfcode{\sphinxupquote{ulsg1}}}{\emph{wert}}{}
Exakte Loesung der Dimension 1
Input:
\begin{quote}
\begin{description}
\item[{wert (float):}] \leavevmode
Werte, auf dem die Funktion ausgewertet wird.

\end{description}
\end{quote}
\begin{description}
\item[{Return:}] \leavevmode\begin{description}
\item[{(float):}] \leavevmode
Wert der Funktion

\end{description}

\end{description}

\end{fulllineitems}

\index{ulsg2() (im Modul aufg\_5\_2)}

\begin{fulllineitems}
\phantomsection\label{\detokenize{index:aufg_5_2.ulsg2}}\pysiglinewithargsret{\sphinxcode{\sphinxupquote{aufg\_5\_2.}}\sphinxbfcode{\sphinxupquote{ulsg2}}}{\emph{wert}}{}
Exakte Loesung der Dimension 2
Input:
\begin{quote}
\begin{description}
\item[{wert (array von float):}] \leavevmode
Werte, auf dem die Funktion ausgewertet wird.

\end{description}
\end{quote}
\begin{description}
\item[{Return:}] \leavevmode\begin{description}
\item[{(float):}] \leavevmode
Wert der Funktion

\end{description}

\end{description}

\end{fulllineitems}

\index{ulsg3() (im Modul aufg\_5\_2)}

\begin{fulllineitems}
\phantomsection\label{\detokenize{index:aufg_5_2.ulsg3}}\pysiglinewithargsret{\sphinxcode{\sphinxupquote{aufg\_5\_2.}}\sphinxbfcode{\sphinxupquote{ulsg3}}}{\emph{wert}}{}
Exakte Loesung der Dimension 3
Input:
\begin{quote}
\begin{description}
\item[{wert (array von float):}] \leavevmode
Werte, auf dem die Funktion ausgewertet wird.

\end{description}
\end{quote}
\begin{description}
\item[{Return:}] \leavevmode\begin{description}
\item[{(float):}] \leavevmode
Wert der Funktion

\end{description}

\end{description}

\end{fulllineitems}



\chapter{Indices and tables}
\label{\detokenize{index:indices-and-tables}}\begin{itemize}
\item {} 
\DUrole{xref,std,std-ref}{genindex}

\item {} 
\DUrole{xref,std,std-ref}{modindex}

\item {} 
\DUrole{xref,std,std-ref}{search}

\end{itemize}


\renewcommand{\indexname}{Python-Modulindex}
\begin{sphinxtheindex}
\def\bigletter#1{{\Large\sffamily#1}\nopagebreak\vspace{1mm}}
\bigletter{a}
\item {\sphinxstyleindexentry{aufg\_5\_2}}\sphinxstyleindexpageref{index:\detokenize{module-aufg_5_2}}
\indexspace
\bigletter{s}
\item {\sphinxstyleindexentry{sparse\_erw}}\sphinxstyleindexpageref{index:\detokenize{module-sparse_erw}}
\end{sphinxtheindex}

\renewcommand{\indexname}{Stichwortverzeichnis}
\printindex
\end{document}